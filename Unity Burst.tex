\subsection{Burst Compiler} \label{burst}
Der Burst Compiler dient der Performance Optimierung von beschränktem Code, weshalb er besser ist als andere Compiler. Dabei sind die Schritte, wie man zu ausführbarem Code kommt,  wie folgt: Zunächst wird der C\# Code von dem Roselyn C\# Compiler in die Intermediate Language (IL) übersetzt. Die IL ist dabei eine Zwischensprache auf die nun verschiedene Compiler angewendet werden können. Normalerweise nutzt Unity die Mono Runtime\footnote{\url{https://docs.unity3d.com/Manual/Mono.html}}. Diese wiederum nutzt bei der Ausführung einen Just-In-Time (JIT) Compiler, welcher die gegebene IL in ausführbaren Code umwandelt, sobald er gebraucht wird. Eine Alternative zu der Mono Runtime ist die IL2CPP Runtime. Diese übersetzt die IL Ahead-Of-Time (AOT), also vor der Ausführung, zunächst in C++ Code. Dann kann der Code von einem C++ Compiler übersetzt und ausgeführt werden.\\
Burst bekommt auch die IL. Diese wird von Burst, welcher den LLVM Compiler\footnote{\url{https://llvm.org/}} nutzt, in ausführbaren Code übersetzt. Der große Vorteil den Burst hierbei hat, ist, dass Burst lediglich beschränkten Code unterstützt. Burst kann beispielsweise nicht mit verwalteten Daten umgehen, dafür braucht es keine Garbage Collection. Burst hat zusätzlichen einen großen Überblick über große Teile des Codes und kann somit beispielsweise auch einen Job mit seiner Implementierung kompilieren. Durch diese Beschränktheit im Programmieren, kann Burst mehr Annahmen über den Code machen und dadurch wesentlich schnelleren Code produzieren. Da man mit dem \textit{ECS} sowieso auf leichtgewichtigen Code wert legt, lässt sich dies gut mit Jobs oder auch ganzen System verbinden. Damit Burst schneller kompilierbaren Code findet, nutzt man das \texttt{[BurstCompile]} Attribut über Jobs, oder Funktionen von Systemen innerhalb des Codes:
\begin{lstlisting}[style=code]
//BurstCompile Attribut
[BurstCompile]
private struct ExampleJob : IJobEntity{}
\end{lstlisting}
\subsubsection{Burst Kompilierung}
Burst hat zwei Arten wie es den Code kompiliert:\\
1. \textbf{Just-In-Time Kompilierung}: Diese Methode wird im Unity Editor verwendet. Das heißt der Compiler kompiliert den Code dann, wenn er verwendet wird. Der Code läuft zunächst mit der normalen Mono Runtime, bis Burst im Hintergrund den Code kompiliert hat. Das bedeutet es wird asynchron kompiliert. Man kann Unity mit dem Attribut \texttt{CompileSynchronously} jedoch auch dazu zwingen, den Code vor dem ersten Ausführen der Coderegion mit Burst zu kompilieren. Dies ist beispielsweise für Laufzeitanalysen sehr vorteilhaft.\\
2. \textbf{Ahead-Of-Time Kompilierung}: Diese Methode wird beim Bauen des Spiels verwendet. Bauen meint hierbei das Spiel von dem Unity Editor in ein ausführbares Programm umzuwandeln. Dabei speichert Burst den kompilierten Code in eine Bibliothek, welche mit dem Spiel ausgeliefert wird. Zur Laufzeit wird dann der kompilierte Code verwendet.
\subsubsection[Beispiel Addition]{Beispiel Addition\protect{\footnote{\url{https://www.youtube.com/watch?v=QkM6zEGFhDY}}}}
Um ein kleines Beispiel zu geben, wie der Burst Compiler Code optimieren kann, gibt es in \hyperref[burstAddition]{Listing \ref*{burstAddition}} eine Beispielmethode, welche zwei Integer addiert und das Ergebnis zurückgibt. Das Beispiel wird einmal mit dem Burst Compiler kompiliert und dann noch einmal mit dem Mono Compiler kompiliert um das Ergebnis zu vergleichen.
\begin{lstlisting}[style=code, caption={[Beispiel einer Addition]Beispiel einer Addition, ausgelagert in eine Methode.}, label = burstAddition]
public static int Add(int ~left~, int @right@){
	return ~left~ + @right@;
}
\end{lstlisting}
Mit Burst kompiliert ergibt sich dieser Assembly Code:
\begin{lstlisting}[style=code, caption={Addition Burst kompiliert}]
Add: (LLVM x64)
	lea eax, [~rcx~, @rdx@]
	ret
\end{lstlisting}
Mono kompiliert die Methode zu diesem Code:
\begin{lstlisting}[style=code, caption={Addition Mono kompiliert}]
Add: (mono 5.16)
	sub		$0x18, %rsp
	mov		~%rdi, (%rsp)~
	mov 	@%rsi, 0x8(%rsp)@
	mov 	~%rdi, %rax~
	add 	@0x8(%rsp)@, ~%eax~
	add 	$0x18, %rsp
	retq
\end{lstlisting}
Wie man sieht gibt es einen großen Unterschied zwischen den kompilierten Ergebnissen. Während beim Mono Compiler die Variablen zunächst ein paar mal verschoben werden, führt der Burst Compiler direkt die Addition (\texttt{lea}) aus und gibt den Wert zurück. Unity Entwickler Alexandre Mutel erwähnt dazu noch, dass dieses Beispiel so in der Realität nicht vorkommt, da man eine Addition nicht in eine Funktion auslagern würde, jedoch trotzdem ein großer Unterschied erkennbar ist (original: \textit{But this particular example would not really happen because most of the time this function is going to be in line by a call at the call site so you won't get this kind of code in Mono, but still the way they are parsing of arguments and so on are still here ...}) \cite{Burst-Unite_LA}.