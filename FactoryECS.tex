\subsection{Datenorientierte Programmierung}
In der datenorientierten Programmierung werden statt der GameObjects \textit{Components} und Systeme entworfen. Dabei werden in der Implementierung des Factory Spiel's alle Aspekte des datenorientierten Technolohgie-Stack's von Unity eingesetzt. Es werden die Daten in \textit{Components} gespeichert, welche von Systemen verarbeitet werden. Die Systeme werden größtenteils so geschrieben, dass Burst verwendet werden kann. Zusätzlich werden Job's erstellt, welche aus Systemen ausgeführt werden.\\
Auch hier zu Beginn das \textit{Component} für ein Item und das dazugehörige System:
\begin{lstlisting}[style=code, caption={Item Component ECS}, label=itemComponent]
public struct ItemComponent : IComponentData
{
    public int2 pos;
    public int itemID;
}
\end{lstlisting}
In dem \textit{Component} wird die Position und die ID des Items gespeichert. Was direkt auffällt ist die klare Trennung der Daten von der Logik, welche sich hier in einem System befindet. Das dazugehörige System sieht anders aus als das MonoBehaviour im Objektorientierten, die Logik ist jedoch dieselbe:
\begin{lstlisting}[style=code, caption={Item System und Job zum Bewegen von Items}]
[BurstCompile(CompileSynchronously = true)]
public partial struct ItemSystem : ISystem
{
    [BurstCompile(CompileSynchronously = true)]
    public void OnUpdate(ref SystemState state)
    {
    	//Job erstellen, Zeit übergeben und schedulen
        new ItemMoveJob
        {
            deltaTime = SystemAPI.Time.DeltaTime
        }.ScheduleParallel();
    }

    [BurstCompile(CompileSynchronously = true)]
    public partial struct ItemMoveJob : IJobEntity
    {
        public float deltaTime;
        
        [BurstCompile(CompileSynchronously = true)]
        private void Execute(ref LocalTransform transform, in ItemComponent item)
        {
        	//Gegenstand wird zu der übergebenen Position pos bewegt
            transform = transform.WithPosition(
                Vector3.Lerp(transform.Position.xyz,
                    new Vector3(item.pos.x, item.pos.y, -0.5f),
                deltaTime * 2f));
        }
    }
}
\end{lstlisting}
Das Item System ist durch das implementierte Interface \texttt{ISystem} als System defniniert und kann daher die \texttt{OnUpdate} Funktion nutzen. Bei dem Item System kommt zusätzlich auch der Burst Compiler und das Job System zum Einsatz. Das Attribut \texttt{BurstCompile} hilft dem Burst Compiler Methoden zu finden, welche mit Burst kompiliert werden sollen. \textit{CompileSynchronously} dient dem testen und besagt, dass erst das System durch den Burst Compiler kompiliert werden muss bevor es zum ersten Mal ausgeführt wird. Andernfalls könnte das System schon laufen, ohne den Burst Compiler genutzt zu haben. Die Methode \texttt{OnUpdate} wird ein mal pro Bild aufgerufen. Sie ist zu vergleichen mit der \texttt{Update} Methode in einem MonoBehaviour. Hier wird der \texttt{IJobEntity} ItemMoveJob erstellt. Dieser Job funktioniert mit einem LocalTransform Component, welches jedes Entity besitzt und die Position angibt, und mit dem ItemComponent aus \hyperref[itemComponent]{Listing \ref{itemComponent}}. Dabei wird das LocalTransform Component zum lesen und schreiben verwendet (erkennbar durch das Keywort ref) und das ItemComponent lediglich zum lesen (erkannbar durch das Keywort in). Auch hier wird, wie in dem Monobehaviour, nun die Position des Items mithilfe der Lerp Funktion von Vector3 und den Daten im ItemComponent verändert. Dieser Job, welcher die tatsächliche Logik für das Item enthält wird in der \texttt{OnUpdate} Methode parallel gescheduled. Dies ist hier speziell sehr vorteilhaft, da es sehr viele Items auf dem Spielfeld geben kann. Dadurch werden nicht tausende Items nacheinander bewegt, sondern alle parallel.\\
Die übergebene Position kommt, wie auch in dem Objektorientierten, von dem Förderband. Auch hier ist die Förderbandlogik in einem Job implementiert. Das hat den Vorteil, dass man dies wieder gut parallelisieren kann. Die Vorgehensweise, dass man die Items von hinten nach vorne ein Förderbandsegment weiterbewegt bleibt jedoch gleich. Ganz anders ist die Methode, wenn man Items erstellen will. Da man mit einem geschedulten Job sich nicht mehr auf dem \textit{Main Thread} befindet, kann man Items nicht mehr direkt erstellen.
\begin{lstlisting}[style=code, caption={Create Item System}]
[BurstCompile(CompileSynchronously = true)]
[UpdateAfter(typeof(ProcessingBuildingSystem))]
public partial struct CreateItemSystem : ISystem
{
	//Lookup um Daten eines anderen Entity auszulesen
    private ComponentLookup<InputConveyorComponent> inputLookup;

    [BurstCompile(CompileSynchronously = true)]
    public void OnCreate(ref SystemState state)
    {
      	//Für die OnUpdate Funktion wird das ItemEntitiesComponent gebraucht
      	//Darin sind alle Items für das Erstellen gespeichert
        state.RequireForUpdate<ItemEntitiesComponent>();
        inputLookup = state.GetComponentLookup<InputConveyorComponent>();
    }

    [BurstCompile(CompileSynchronously = true)]
    public void OnUpdate(ref SystemState state)
    {
        inputLookup.Update(ref state);
        var ecbSingleton = SystemAPI.GetSingleton<BeginSimulationEntityCommandBufferSystem.Singleton>();
        //Entity Command Buffer wird erstellt
        var ecb = ecbSingleton.CreateCommandBuffer(state.WorldUnmanaged);
        var itemEntities = SystemAPI.GetSingleton<ItemEntitiesComponent>();
        // Job wird erstellt und gescheduled
        new CreateItemJob
        {
            inputLookup = inputLookup,
            ecb = ecb,
            itemEntities = itemEntities
        }.Schedule();
        state.CompleteDependency();
    }
}
\end{lstlisting}
Hier sieht man eine Besonderheit des \textit{Entity Component Systems}, der \textit{Entity Command Buffer} (ECB). Dadurch dass strukturelle Änderungen nur auf dem \textit{Main Thread} passieren dürfen braucht man einen ECB um diese Änderungen zu sammeln und an späteren Stelle auf dem \textit{Main Thread} auszuführem. Hier wird der ECB verwendet um strukturelle Änderungen aus einem Job vorzunehmen.
\begin{lstlisting}[style=code, caption={Create Item Job}]
[BurstCompile(CompileSynchronously = true)]
[WithNone(typeof(OutputNotFoundTag))]
public partial struct CreateItemJob : IJobEntity
{
    public EntityCommandBuffer ecb;
    //Wenn nur von gelesen wird kann ReadOnly verwendet werden
    [ReadOnly] public ComponentLookup<InputConveyorComponent> inputLookup;
    [ReadOnly] public ItemEntitiesComponent itemEntities;

    [BurstCompile(CompileSynchronously = true)]
    //Job arbeitet mit dem Output eines Gebäudes
    private void Execute(ref OutputProcessingBuildingComponent output)
    {
        //Wenn der Output leer ist gibt es nichts zu tun
        if(output.itemID == -1) return;
        var itemID = output.itemID;
        var input = inputLookup[output.outputEntity];
        //Wenn Input des Förderbandes belegt ist wird auch keine
        //Änderung vorgenommen
        if(input.occupied || input.item != Entity.Null) return;
        //Item wird aus dem Output entfernt und mittels ECB
        //wird ein neues Item erstellt 
        output.itemID = -1;
        output.itemCreated = true;
        var itemEntity = itemEntities.GetEntityWithID(itemID);
        var item = ecb.Instantiate(itemEntity);
        //Position und Item Component des Items werden gesetzt
        ecb.SetComponent(item, LocalTransform.FromPositionRotationScale(
            new float3(output.pos.x, output.pos.y, -0.5f),
            quaternion.identity, 0.5f));
        ecb.SetComponent(item,
            new ItemComponent{pos = output.pos, itemID = itemID});
        //Input wird das Item zugewiesen
        ecb.SetComponent(output.outputEntity,
            new InputConveyorComponent{
            item = item, pos = input.pos, occupied = true});
    }
}
\end{lstlisting}
Der \textit{Entity Command Buffer} erstellt das Item und ändert noch in den \textit{Components} die Position des Items. Zusätzlich wird dem Input des Förderbandes das erstellte Item zugewiesen. Diese Änderungen werden in dem Job aufgenommen und nach dem Job in der richtigen Reihenfolge angewendet.