\subsection{Datenorientierte Programmierung}
In der Datenorientierten Programmierung werden statt der GameObjects \textit{Components} und Systeme entworfen. Dabei werden in der Implementierung des Factory Spiels alle Aspekte des \textit{Datenorientierten Technologie-Stack's} von Unity eingesetzt. Es werden die Daten in \textit{Components} gespeichert, welche von Systemen verarbeitet werden. Die Systeme werden, soweit möglich, so umgesetzt, dass der Burst Compiler verwendet werden kann. Zusätzlich werden Jobs erstellt, welche aus Systemen ausgeführt werden.\\
Das Item \textit{Component} des datenorientierten Ansatzes zeigt \hyperref[lstItemComponent]{Listing \ref*{lstItemComponent}}.
\begin{lstlisting}[style=code, caption={[Item \textit{Component} im datenorientierten Ansatz]Item \textit{Component} im datenorientierten Ansatz. Es werden nur die Daten des Items gespeichert. Die Logik des Items befindet sich in einem System.}, label=lstItemComponent]
public struct ItemComponent : IComponentData
{
    public int2 pos;
    public int itemID;
}
\end{lstlisting}
In dem \texttt{ItemComponent} wird die Position und die ID des Items gespeichert. Was direkt auffällt, ist die klare Trennung der Daten von der Logik, welche sich hier in einem System befindet. Die Logik des objektorientierten Items (\hyperref[lstItemKomponente]{Listing \ref*{lstItemKomponente}}) befindet sich im datenorientierten Ansatz in dem \texttt{ItemSystem} in \hyperref[lstItemSystem]{Listing \ref*{lstItemSystem}}.
\begin{lstlisting}[style=code, caption={[Item System im datenorientierten Ansatz.]Item System zum Bewegen von Items im datenorientierten Ansatz. Aus der \texttt{OnUpdate} Methode wird ein Job geschedult, welcher die Items bewegt.}, label=lstItemSystem]
[BurstCompile(CompileSynchronously = true)]
public partial struct ItemSystem : ISystem
{
    [BurstCompile(CompileSynchronously = true)]
    public void OnUpdate(ref SystemState state)
    {
    	//Job erstellen, Zeit übergeben und schedulen
        new ItemMoveJob
        {
            deltaTime = SystemAPI.Time.DeltaTime
        }.ScheduleParallel();
    }

    [BurstCompile(CompileSynchronously = true)]
    public partial struct ItemMoveJob : IJobEntity
    {
        public float deltaTime;
        
        [BurstCompile(CompileSynchronously = true)]
        private void Execute(ref LocalTransform transform,
            in ItemComponent item)
        {
        	//Gegenstand wird zu der übergebenen Position pos bewegt
            transform = transform.WithPosition(
                Vector3.Lerp(transform.Position.xyz,
                    new Vector3(item.pos.x, item.pos.y, -0.5f),
                deltaTime * 2f));
        }
    }
}
\end{lstlisting}
Das \texttt{ItemSystem} ist durch das implementierte Interface \texttt{ISystem} als System defniniert und kann daher die \texttt{OnUpdate} Methode nutzen. Bei dem Item System kommt zusätzlich auch der Burst Compiler und das Job System zum Einsatz. Das Attribut \texttt{BurstCompile} hilft dem Burst Compiler Methoden zu finden, welche mit Burst kompiliert werden sollen. \texttt{CompileSynchronously} dient dem Testen und besagt, dass erst das System durch den Burst Compiler kompiliert werden muss, bevor es zum ersten Mal ausgeführt wird. Andernfalls könnte das System schon laufen, ohne von dem Burst Compiler profitiert zu haben. Die Methode \texttt{OnUpdate} wird ein mal pro Bild aufgerufen (Zeile 5). Sie ist zu vergleichen mit der \texttt{Update} Methode in einem \texttt{MonoBehaviour}. In der \texttt{OnUpdate} Methode wird der \texttt{ItemMoveJob} erstellt (Zeile 8 - 11). Dieser Job funktioniert mit dem \texttt{ItemComponent} und \texttt{LocalTransform} \textit{Component} eines \textit{Entities}. Das \texttt{LocalTransform} \textit{Component} gibt die Position des \textit{Entities} an. Das \texttt{ItemComponent} \textit{Component} ist in \hyperref[lstItemComponent]{Listing \ref{lstItemComponent}} definiert und speichert die Daten des Items. In dem Job wird das \texttt{LocalTransform} \textit{Component} zum Lesen und Schreiben verwendet (erkennbar durch das Keywort \texttt{ref}) und das \texttt{ItemComponent} lediglich zum Lesen (erkennbar durch das Keywort \texttt{in}) (Zeile 20 - 21). Auch hier wird, wie in dem \texttt{MonoBehaviour}, nun die Position des Items mithilfe der \texttt{Lerp} Funktion von \texttt{Vector3} und den Daten im \texttt{ItemComponent} verändert (Zeile 24 - 26). Dieser Job, welcher die tatsächliche Logik für das Item enthält, wird in der \texttt{OnUpdate} Methode parallel gescheduled (Zeile 11). Dies ist hier speziell sehr vorteilhaft, da es sehr viele Items auf dem Spielfeld geben kann. Dadurch werden nicht tausende Items nacheinander bewegt, sondern alle parallel.\\
Die übergebene Position kommt, wie auch in dem objektorientierten Ansatz, von dem Förderband. Auch hier ist die Förderbandlogik in einem Job implementiert. Die Vorgehensweise, dass man die Items von hinten nach vorne ein Förderbandsegment weiterbewegt bleibt jedoch gleich. Ganz anders ist die Funktionsweise, wenn man Items erstellen will. Da man sich mit einem geschedulten Job nicht mehr auf dem \textit{Main Thread} befindet, kann man Items nicht mehr direkt erstellen.
\begin{lstlisting}[style=code, caption={[Erstellung eines Items im datenorientierten Ansatz.]Erstellung eines Items im datenorientierten Ansatz. Zur Erstellung eines Items aus einem Job, wird ein \textit{Entity Command Buffer} verwendet.}]
[BurstCompile(CompileSynchronously = true)]
[UpdateAfter(typeof(ProcessingBuildingSystem))]
public partial struct CreateItemSystem : ISystem {
    //Lookup um Daten eines anderen Entity auszulesen
    private ComponentLookup<InputConveyorComponent> inputLookup;

    [BurstCompile(CompileSynchronously = true)]
    public void OnCreate(ref SystemState state) {
      	//Für die OnUpdate Funktion wird das ItemEntitiesComponent gebraucht
      	//Darin sind alle Items für das Erstellen gespeichert
        state.RequireForUpdate<ItemEntitiesComponent>();
        inputLookup = state.GetComponentLookup<InputConveyorComponent>();
    }
    
    [BurstCompile(CompileSynchronously = true)]
    public void OnUpdate(ref SystemState state) {
        inputLookup.Update(ref state);
        var ecbSingleton = SystemAPI.GetSingleton
            <BeginSimulationEntityCommandBufferSystem.Singleton>();
        //Entity Command Buffer wird erstellt
        var ecb = ecbSingleton.CreateCommandBuffer(state.WorldUnmanaged);
        var itemEntities = SystemAPI.GetSingleton<ItemEntitiesComponent>();
        // Job wird erstellt und gescheduled
        new CreateItemJob {
            inputLookup = inputLookup,
            ecb = ecb,
            itemEntities = itemEntities
        }.Schedule();
        state.CompleteDependency();
    }
}
\end{lstlisting}
Hier sieht man eine Besonderheit des \textit{Entity Component Systems}, der \textit{Entity Command Buffer} (\textit{ECB}). Dadurch, dass strukturelle Änderungen nur auf dem \textit{Main Thread} passieren dürfen, braucht man einen \textit{ECB} um diese Änderungen zu sammeln und an späteren Stelle auf dem \textit{Main Thread} auszuführen. Dafür wird ein \textit{ECB} erstellt (Zeile 21) und dieser dem Job übergeben (Zeile 26). \hyperref[lstCreateItemJob]{Listing \ref{lstCreateItemJob}} zeigt den Job, der diese strukturellen Änderungen vornimmt.
\begin{lstlisting}[style=code, caption={[Job in dem ein \textit{ECB} verwendet wird]Job in dem ein \textit{ECB} verwendet wird. Der \textit{ECB} sammelt die Änderungen aus dem Job und spielt sie nach dem Job wieder ab.}, label=lstCreateItemJob]
[BurstCompile(CompileSynchronously = true)]
[WithNone(typeof(OutputNotFoundTag))]
public partial struct CreateItemJob : IJobEntity {
    public EntityCommandBuffer ecb;
    //Wenn nur von gelesen wird kann ReadOnly verwendet werden
    [ReadOnly] public ComponentLookup<InputConveyorComponent> inputLookup;
    [ReadOnly] public ItemEntitiesComponent itemEntities;

    [BurstCompile(CompileSynchronously = true)]
    private void Execute(ref OutputProcessingBuildingComponent output) {
        //Wenn die Ausgabe leer ist gibt es nichts zu tun
        if(output.itemID == -1) return;
        var itemID = output.itemID;
        var input = inputLookup[output.outputEntity];
        //Wenn die Eingabe des Förderbandes belegt ist wird auch keine
        //Änderung vorgenommen
        if(input.occupied || input.item != Entity.Null) return;
        //Item wird aus der Ausgabe entfernt und mittels ECB
        //wird ein neues Item erstellt 
        output.itemID = -1;
        output.itemCreated = true;
        var itemEntity = itemEntities.GetEntityWithID(itemID);
        var item = ecb.Instantiate(itemEntity);
        //Position und Item Component des Items werden gesetzt
        ecb.SetComponent(item, LocalTransform.FromPositionRotationScale(
            new float3(output.pos.x, output.pos.y, -0.5f),
            quaternion.identity, 0.5f));
        ecb.SetComponent(item,
            new ItemComponent{pos = output.pos, itemID = itemID});
        //Der Eingabe wird das Item zugewiesen
        ecb.SetComponent(output.outputEntity,
            new InputConveyorComponent {
            item = item, pos = input.pos, occupied = true});
    }
}
\end{lstlisting}
Der \textit{ECB} erstellt das Item (Zeile 23) und ändert noch in den \textit{Components} die Position des Items (Zeile 25 - 29). Zusätzlich wird der Eingabe des Förderbandes das erstellte Item zugewiesen (Zeile 31 - 33). Diese Änderungen werden in dem Job aufgenommen und nach dem Job in der richtigen Reihenfolge angewendet.