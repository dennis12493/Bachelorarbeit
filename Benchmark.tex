\section{Benchmark}
Um die Performance von dem in \hyperref[sec:factory]{Kapitel \ref{sec:factory}} vorgestellten Videospiel zu testen, wird dieses gebenchmarkt. Ein Benchmark ist ein Maßstab um Leistungen zu vergleichen. Hier wird das datenorientierte Spiel (im Folgenden \textit{ECS} genannt) mit dem objektorientierten Spiel (im Folgenden Mono genannt) verglichen. Der Unterschied in der Leistungsfähigkeit zwischen \textit{ECS} und Mono ist besonders interessant, weshalb in diesem Benchmark die Zeit gemessen wird, die ein Bild zur Berechnung gebraucht hat. Eine andere, oft verwendete Einheit, sind die durchschnittlichen Bilder pro Sekunde (fps).\\
In diesem Kapitel wird nun der direkte Vergleich von \textit{ECS} und Mono angestellt. Dazu stellt Unity einige hilfreiche Tools zu Verfügung, um Aspekte in einem Spiel zu messen. Auch kann dadurch gut nachvollzogen werden, wie lange eine Aufgabe, wie zum Beispiel die Ausführung der \texttt{Update} Funktionen, benötigt hat. Bei den nachfolgenden Tests wird jeweils das implementierte Spiel aus \hyperref[sec:factory]{Kapitel \ref*{sec:factory}} verwendet. Bei dem Benchmark werden immer 2000 Bilder aufgezeichnet. Sobald ein Stahlbarren hinten in dem Kasten angekommen ist (\hyperref[fig:steel]{Abbildung \ref*{fig:steel}}), wird das Spiel pausiert. Die letzten 2000 Bilder werden dann zur Auswertung verwendet. Zusätzlich werden mehrere Durchläufe gemessen. Angefangen wird mit 1, 10 und 100 Fabriken. In 100er Schritten wird dann die Anzahl der Fabriken erhöht, bis zu 1000 Fabriken.\\
Die Benchmark Tests werden auf einer Maschine mit einer NVIDIA GeForce RTX 3070, einem Intel Core i7 12700 und 32 Gb DDR 4 RAM ausgeführt. Die Tests liefen außerdem auf einem Windows 11 Betriebssystem.
\subsection{\textit{Profiler}}
Der Unity \textit{Profiler} ist ein Tool, mit dem man Performance Informationen über sein Spiel bekommt. Mit dem \textit{Profiler} kann man sein Spiel aufzeichnen und Daten über eine festgelegte Anzahl von Bildern sammeln. Dazu gehören die CPU Auslastung, der Speicherverbrauch, wie viele Objekte gezeichnet werden und noch vieles mehr. Während das Spiel ausgeführt wird, werden diese Daten live angezeigt. Zusätzlich lassen sich die Daten auch beim Pausieren des Spiel abspeichern und andere vorher gespeicherte Daten wieder laden um sie zu betrachten. \hyperref[fig:profiler]{Abbildung \ref*{fig:profiler}} zeigt den \textit{Profiler} mit schon aufgezeichnete Daten. 
\begin{figure}[H]
\centering
\begin{annotatedFigure}
	{\includegraphics[scale=0.428]{Bilder/Profiler.png}}
    \annotatedFigureBox{0.1688,0.9496}{0.363,1.01}{A}{0.363,0.9496}
	\annotatedFigureBox{0.12,0.65}{1.005,0.76}{B}{1.005,0.65}
	\annotatedFigureBox{-0.005,0.3778}{0.1259,1.01}{C}{0.1259,0.3778}
	\annotatedFigureBox{0.38,0.0888}{0.8705,0.3831}{D}{0.8705,0.0888}
\end{annotatedFigure}
\caption[Der \textit{Profiler} im Unity Editor]{Der \textit{Profiler} im Unity Editor. Mit ihm lassen sich die Auslastung des Spiels während des Spielens aufzeichnen und die Daten zur weiteren Verwendung abspeichern. In der oberen Leiste (A), lässt sich das Spiel wie gewünscht aufzeichnen und man sieht die Anzahl an aufgezeichneten Bildern. Man kann auch einzelne Bilder zur weiteren Inspektion auswählen. Darunter (B) wird der Verlauf der aufgezeichneten Bilder dargestellt. Hier sieht man für jedes Modul eine Auslastung über den Verlauf der 2000 Bilder. Links (C) werden die einzelnen Module angezeigt. Sichtbar sind hier die Module CPU Usage, Rendering und Memory, wobei man für weitere im Editor nach unten scrollen muss. Unten (D) ist eine genauere Timeline sichtbar, da hier nur einige Bilder abgebildet werden.}
\label{fig:profiler}
\end{figure}
In der \hyperref[fig:profiler]{Abbildung \ref*{fig:profiler}} erkennt man die einzelnen Auslastungen und ihre Kategorien. Zusätzlich kann man ein bestimmtes Bild auswählen und erhält dazu weitere Details in der Timeline weiter unten. Diese aufgezeichneten Daten lassen sich dann in dem \textit{Profile Analyzer} weiter analysieren.
\subsection{\textit{Profile Analyzer}}
Der \textit{Profile Analyzer} dient der weiteren Analyse von aufgenommenen Profilen mit dem \textit{Profiler}. Zusätzlich hat man mit dem \textit{Profile Analyzer} die Möglichkeit zwei Profile direkt miteinander zu vergleichen. \hyperref[fig:profile_analyzer]{Abbildung \ref*{fig:profile_analyzer}} zeigt den Vergleich zweier Profile.
\begin{figure}[H]
\centering
\begin{annotatedFigure}
	{\includegraphics[scale=0.31]{Bilder/Profile Analyzer.png}}
	\annotatedFigureBox{-0.005,0.8046}{0.182,1.01}{A}{0.182,0.8046}
	\annotatedFigureBox{0.86,-0.01}{1.005,1.01}{D}{1.005,-0.01}
	\annotatedFigureBox{-0.005,0.4842}{0.857,0.5942}{B}{0.857,0.4842}
	\annotatedFigureBox{-0.005,-0.01}{0.445,0.4804}{C}{0.445,-0.01}
\end{annotatedFigure}
\caption[Der \textit{Profile Analyzer} im Unity Editor]{Der \textit{Profile Analyzer} im Unity Editor. Es werden die Mono- und die \textit{ECS} Messdaten des Spiels miteinander verglichen. Links oben (A) erkennt man, welche Daten man gerade vergleicht und sieht dazu eine Timeline beider Daten. Die \textit{ECS} Daten sind im Bild blau dargestellt, die Mono Daten rot. In der Mitte des Bildes (B) hat man den Vergleich des Median Bildes. Deutlich zu erkennen ist die wesentlich kürzere Laufzeit des \textit{ECS} Bildes. In dem unteren Teil (C) werden wichtige Spielfunktionen im Vergleich dargestellt. Beispielsweise ist ganz oben der \textit{PlayerLoop} (gesamte Berechnungszeit eines Bildes). Dazu erkennt man an dem Balken, welches der Spiele eine höhere Berechnungszeit hatte. Rechts (D) gibt es zusätzlich noch eine Zusammenfassung über die gesamte Anzahl an Bildern und Threads in beiden Spielen.}
\label{fig:profile_analyzer}
\end{figure}
Oberhalb erkennt man die zu vergleichenden Daten und wie viele Bilder die beiden Daten enthalten. Hier ist erkennbar, dass einmal die Daten des \textit{ECS} mit 500 Fabriken (blau dargestellt) und des Mono's mit 500 Fabriken (rot dargestellt) verglichen werden. Die wichtigen Teile sind hierbei die zusammenfassende Übersicht (rechts), welche eine Übersicht über alle Bilder oder Threads bietet, und den Vergleich von Kernfunktionen der Spiele, wie beispielsweise der \texttt{Update} Funktionen und dem Erstellen, oder Zerstören von Objekten. Das Problem hierbei ist jedoch, dass manche der Kernfunktionen von dem objektorientierten Ansatz anders heißen als die vom datenorientierten Ansatz und man sie somit schlechter gegenüberstellen kann. Dennoch kann man die Zusammenfassung der Bilder gut nutzen, um die Spiele zu vergleichen.
\newpage
\subsection{Vergleich}
\hyperref[fig:benchmark]{Abbildung \ref*{fig:benchmark}} zeigt den direkten Vergleich der Laufzeiten von Monobehaviour zu dem \textit{ECS}. Dabei ist Mono rot und \textit{ECS} blau markiert.
\begin{figure}[H]
\centering
\includegraphics[scale=0.8]{Bilder/Benchmark.png}
\caption[Der Vergleich von Mono und dem \textit{Entity Component System}]{Der Vergleich von Mono und dem \textit{Entity Component System}. Es werden die durchschnittlichen Millisekunden, die ein Bild zur Berechnung gebraucht hat, für verschiedene Anzahlen von Fabriken, angezeigt.}
\label{fig:benchmark}
\end{figure}
\begin{table}[h]
\centering
\begin{tabular}{c|c|c|c|c|c|c}
 & \multicolumn{3}{c|}{\textit{ECS}} & \multicolumn{3}{c}{Mono}\\
Anzahl Fabriken & Min & Mean & Max & Min & Mean & Max\\
\hline
1 & 2.42 ms & 3.92 ms & 10.94 ms & 2.77 ms & 3.19 ms & 8.65 ms\\
10 & 2.46 ms & 3.96 ms & 10.47 ms & 3.0 ms & 3.7 ms & 9.33 ms\\
100 & 3.57 ms & 3.96 ms & 9.03 ms & 5.27 ms & 6.24 ms & 13.08 ms\\
200 & 2.76 ms & 4.25 ms & 10.87 ms & 7.87 ms & 9.27 ms & 19.16 ms\\
300 & 3.94 ms & 4.36 ms & 9.37 ms & 10.44 ms & 12.44 ms & 25.37 ms\\
400 & 4.18 ms & 4.6 ms & 9.82 ms & 10.61 ms & 15.03 ms & 33.69 ms\\
500 & 4.3 ms & 4.76 ms & 9.81 ms & 11.12 ms & 17.58 ms & 39.04 ms\\
600 & 4.52 ms & 5.01 ms & 10.56 ms & 10.66 ms & 21.14 ms & 61.65 ms\\
700 & 4.48 ms & 5.21 ms & 10.16 ms & 9.33 ms & 21.49 ms & 53.76 ms\\
800 & 4.48 ms & 5.39 ms & 10.89 ms & 8.51 ms & 23.11 ms & 65.15 ms\\
900 & 4.35 ms & 5.6 ms & 10.6 ms & 8.86 ms & 28.22 ms & 76.37 ms\\
1000 & 4.6 ms & 5.74 ms & 10.99 ms & 9.65 ms & 34.07 ms & 90.31 ms\\
\end{tabular}\\
\caption{Die genauen Daten des Benchmarks. Es werden das Minimum, Maximum und die durchschnittlichen Berechnungszeiten für alle Fabrikanzahlen gezeigt.}
\label{table}
\end{table}
Ganz deutlich erkennbar schneidet Mono wesentlich schlechter ab. Mono braucht deutlich länger um ein Bild zu verarbeiten, vor allem wenn sich die Anzahl der Fabriken erhöht. Dies bestätigt die Annahme, dass das \textit{ECS} wesentlich besser darin ist große Mengen an Objekten zu verarbeiten. Auch hier steigt die durchschnittliche Zeit ein Bild zu verarbeiten, jedoch nur sehr gering. Eine Ausnahme bildet hier jedoch eine sehr geringe Anzahl an Fabriken. Vor allem bei einer Fabrik ist Mono schneller als das \textit{ECS}. \hyperref[table]{Tabelle \ref*{table}} zeigt dazu genauere Werte. Das liegt an dem Overhead den das \textit{ECS} mit sich bringt. Für eine geringe Anzahl an \textit{Entities} ist es ein großer Overhead, Jobs zu erstellen und diese auf mehreren \textit{Threads} auszuführen. Hier wäre es schneller die Aktionen direkt auf dem \textit{Main Thread} auszuführen, statt einen Job dafür zu schedulen.