\section{Factory Spiel in Unity}\label{sec:factory}
Um zu sehen, welches Potential die datenorientierte Programmierung in der Spieleentwicklung bietet wurde eine Spielsimulation sowohl datenorientiert als auch objektorientiert erstellt. Das Spiel, welches simuliert und gemessen wird ist eine Art Aufbauspiel. Verglichen kann es mit dem populären Spiel Factorio\footnote{https://www.factorio.com/}. Hier wird es jedoch etwas schlichter und einfacher gehalten. Um beide Programmierparadigmen miteinander zu vergleichen wird eine kleine Fabrik in die Spielwelt generiert und vervielfältigt. Dies soll den Spieler simulieren. In Abbildung \ref{fig:steel} sieht man wie eine Produktionsstraße für Stahl in dem Spiel aussehen kann:
\begin{figure}[H]
\includegraphics[scale=0.87]{Bilder/Stahl Fabrik.png}
\caption{Eine Stahl Fabrik in der Unity Spielesimulation.}
\label{fig:steel}
\end{figure}
Ganz links befinden sich zwei Kästen. Diese sollen Erzbohrer darstellen welche Eisenerz (silbern) und Kohle (braun) aus der Erde befördern und rechts auf ein Förderband legen. Die beiden Förderbänder transportieren die Kohle und das Eisenerz nach rechts weiter. Das Eisenerz wird als nächstes in Eisenbarren geschmolzen und sind deshalb rechteckig. Die Kohle und die Eisenbarren kommen dann gemeinsam in den großen Würfel, wo sie zu Stahl verarbeitet werden. Dieser Stahl wird dann wieder auf ein Förderband gelegt und in dem kleinen Würfel anschließend gelöscht. Die ganze Produktionskette soll möglichst einem Spiel nahekommen damit es eine möglichst realistische Simulation bietet und die Realität widerspiegelt. Nachfolgend wird gezeigt, wie die Spielsimulation möglichst ähnlich umgesetzt und gebenchmarkt wird.
\subsection{Objektorientierte Programmierung}
In der Objektorientierten Programmierung mit Unity dreht sich alles um \textit{GameObjects}. Jedes Objekt in der Szene, egal ob sichtbar oder nicht, ist ein \textit{GameObject}. \textit{GameObjects} können verschiedene Komponenten haben, welche Eigenschaften definieren, also Daten speichern. Diese Komponenten sind jedoch nicht zu verwechseln mit den \textit{Components} aus dem datenorientierten Ansatz von Unity. Die Komponenten im objektorientierten Ansatz haben jeweils eine \texttt{Start} und eine \texttt{Update} Methode welche das Verhalten definieren. Die \texttt{Start} Methode wird einmalig vor dem ersten Aufruf der \texttt{Update} Methode ausgeführt. Die \texttt{Update} Methode läuft einmal pro ausgegebenem Bild. \hyperref[lstItemKomponente]{Listing \ref*{lstItemKomponente}} zeigt die Item Komponente des Item \textit{GameObjects}.
\begin{lstlisting}[style=code, caption={[Item Komponente im objektorientierten Ansatz]Item Komponente im objektorientierten Ansatz. Es speichert die Position und seine ID. In der \texttt{Update} Methode bewegt sich das Item linear zu der übergebenen Position.}, label=lstItemKomponente]
public class Item : MonoBehaviour {
    private int2 pos;
    //Serialisiertes Feld für den Unity Editor
    [SerializeField] private int itemID;

    void Update() {
    	//Gegenstand wird zu der übergebenen Position pos bewegt
        transform.position = Vector3.Lerp(transform.position,
            new Vector3(pos.x, pos.y, -0.5f), Time.deltaTime * 2f);
    }

    public void SetPosition(int2 pos) {
        this.pos = pos;
    }
}
\end{lstlisting}
Wie man sieht, beinhaltet das \texttt{MonoBehaviour} nicht nur die Daten, sondern auch die Logik. Für ein Item benötigen wir zum einen die Position, wohin sich das Item bewegen soll, zum anderen speichern wir auch eine ID über die wir das Item ganz einfach identifizieren können. Das Attribut \texttt{SerializeField} (Zeile 5) zwingt Unity dazu, ein editierbares Feld im Editor zu erstellen an dem man die \texttt{itemID} setzen kann. Dadurch lassen sich vorgefertigte Items erstellen, welche unterschiedlich aussehen und man zusätzlich die Item ID setzen kann. Beispielsweise ist ein Eisenbarren quadratisch, hat ein silbernes Material und bekommt die ID drei. Diese vorgefertigten Items kann man dann zur Laufzeit erstellen.
\begin{figure}[H]
\centering
\includegraphics[scale=1]{Bilder/SerializeField.png}
\caption[\texttt{SerializeField}: Ein editierbares Feld in dem Unity Editor]{Ein editierbares Feld in dem Unity Editor. Das Feld wird durch ein \texttt{public} Attribut, oder dem \texttt{SerializeField} Flag erzeugt.}
\label{fig:SerializeField}
\end{figure}
Die \texttt{Start} Methode ist in diesem Fall nicht notwendig. Die \texttt{Update} Methode bewegt das Item langsam an die übergebene Position. \texttt{Time.deltaTime} (Zeile 11) gibt die Zeit in Sekunden von dem letzten Bild bis zu dem momentanen Bild an. Dadurch wird die Bewegung linear.\\
Ein weiterer Teil der Spielesimulation ist die Bewegung der Items über das Förderband. Das Förderband übergibt die Position an das Item und bestimmt daher, in welche Richtung sich das Item bewegen soll. \hyperref[lstConveyorKomponente]{Listing \ref*{lstConveyorKomponente}} beschreibt den Aufbau des Förderbandes.
\begin{lstlisting}[style=code, caption={[Förderband Komponente im objektorientierten Ansatz]Förderband Komponente im objektorientierten Ansatz. Es speichert die Ein- und Ausgabe des Förderbandes und alle Segmente die zu dem Förderband gehören. Das Förderband sorgt dafür, dass alle Items entlang des Förderbandes weiterbewegt werden.}, label=lstConveyorKomponente]
public class BeltPath : MonoBehaviour
{
    private List<ConveyorComponent> beltPath = new ();
    //Referenzen auf die Komponenten des GameObjects werden geholt
    private InputConveyorComponent input 
        = GetComponent<InputConveyorComponent>();
    private OutputConveyorComponent output 
        = GetComponent<OutputConveyorComponent>();
    private float timeToMove = 2f;

    public void Update()
    {
        timeToMove -= Time.deltaTime;
        //Bewegung der Items wird alle 2 Sekunden durchgeführt
        if(timeToMove > 0) return;
        var lastBelt = beltPath[^1];
        if (!ReferenceEquals(lastBelt.item, null)
            && ReferenceEquals(output.GetItem(), null)) {
            //Gegenstand von dem letzten Förderbandsegment
            //auf die Ausgabe legen
            var item = lastBelt.item;
            var itemComponent = item.GetComponent<Item>();
            itemComponent.SetPosition(output.GetPosition());
            output.SetItem(item);
            lastBelt.item = null;
        }
        //Gegenstände von hinten nach vorne um ein Segment verschieben
        for (int i = beltPath.Count-2; i >= 0; i--) {
            var thisConveyor = beltPath[i];
            var lastConveyor = beltPath[i + 1];
            if (!ReferenceEquals(thisConveyor.item, null)) {
                if (ReferenceEquals(lastConveyor.item, null)) {
                    //Item kann verschoben werden
                    var item = thisConveyor.item;
                    var itemComponent = item.GetComponent<Item>();
                    lastConveyor.item = item;
                    //Position des Items aktualisieren
                    itemComponent.SetPosition(lastConveyor.pos);
                    thisConveyor.item = null;
                }
            }
        }
        var firstConveyor = beltPath[0];
        input.SetOccupied(!ReferenceEquals(firstConveyor.item, null));
        if (!ReferenceEquals(input.GetItem(), null)
            && ReferenceEquals(firstConveyor.item, null)) {
            //Item wird von der Eingabe auf das erste Segment gelegt
            firstConveyor.item = input.GetItem();
            input.RemoveItem();
        }
        //Zeit wird wieder hochgestellt
        timeToMove += 2f;
    }
}
\end{lstlisting}
Für das Förderband wird eine Liste mit vorhandenen Segmenten, die Eingabe, die Ausgabe und eine Zeit gespeichert (Zeile 3 - 8). Die Zeit wird in der \texttt{Start} Methode initialisiert und die Ein- beziehungsweise Ausgabe wird über die Funktion \texttt{GetComponent} von dem \textit{GameObject} geholt. In der \texttt{Update} Funktion wird zunächst nur die \texttt{timeToMove} Variable herunter gezählt (Zeile 13). Sollte diese Variable unter Null fallen, werden alle Items von hinten nach vorne ein Segment weiter bewegt, sofern dies möglich ist. Ist die Ausgabe nicht belegt, wird ein vorhandenes Item in die Ausgabe gelegt (Zeile 17 - 26). In der Schleife werden die Items auf den einzelnen Segmenten weiterbewegt (Zeile 28 - 42) und wenn ein Item in der Eingabe liegt wird dieses auf das erste Segment weiterbewegt (Zeile 45 - 50). Immer wenn ein Item weitergegeben wird (egal ob an ein Segment, oder an die Ausgabe) wird auch die neue Position an das Item weitergegeben (Zeile 38). Durch das Bewegen der Items von hinten nach vorne verhindert man, dass sich Items nicht bewegen, obwohl sie es könnten.\\
Auf den Förderbändern sind Items sichtbare \textit{GameObjects}. In Gebäuden wird lediglich mit den ID's der Items gearbeitet, da man hier keine Objekte braucht. Items werden in der Ausgabe eines Gebäudes erstellt und von dort an das Förderband übergeben.  Die \texttt{Update} Methdode dafür zeigt \hyperref[lstCreateItemOOP]{Listing \ref*{lstCreateItemOOP}}.
\begin{lstlisting}[style=code, caption={[Erstellung eines Items im objektorientierten Ansatz]Erstellung eines Items im objektorientierten Ansatz. Wenn ein Förderband verbunden ist und ein Item erstellt werden soll, wird das passende \textit{GameObject} instanziert. Dieses wird dann an das Förderband übergeben.}, label=lstCreateItemOOP]
void Update()
{
    //Wenn die Ausgabe leer ist gibt es nichts zu tun
    if(itemID == -1) return;
    outputGameObject ??= BuildingDictionary.Instance
        .GetGameObjectAtPosition(pos);
    if (ReferenceEquals(outputGameObject, null) ||
        !outputGameObject.TryGetComponent(out InputConveyorComponent input))
        return;
    //Wenn die Eingabe des Förderbandes belegt ist wird auch keine
    //Änderung vorgenommen
    if(input.IsOccupied() || !ReferenceEquals(input.GetItem(), null))
        return;
    var itemGameObject = Items.INSTANCE.GetItem(itemID);
    //Item wird erstellt
    var item = Instantiate(itemGameObject, new Vector3(pos.x, pos.y, -0.5f), Quaternion.identity);
    item.transform.localScale = new Vector3(0.5f, 0.5f, 0.5f);
    var itemComponent = item.GetComponent<Item>();
    itemComponent.SetPosition(pos);
    var inputConveyorComponent = outputGameObject
        .GetComponent<InputConveyorComponent>();
    //Der Eingabe wird das Item zugewiesen
    inputConveyorComponent.SetItem(item);
    //Item wird aus der Ausgabe entfernt
    itemID = -1;
    itemCreated = true;
}
\end{lstlisting}
Da man sich im objektorientierten Ansatz, ohne weitere Vorkehrungen, immer auf dem \textit{Main Thread} befindet, lassen sich die Items direkt erstellen und der Eingabe übergeben. Das Zerstören von Items, also der Fall wenn Items von einem Förderband in ein Gebäude übergeben werden, funktioniert sehr ähnlich zu dem Erstellen von Items.
\subsection{Datenorientierte Programmierung}
In der datenorientierten Programmierung werden statt der GameObjects \textit{Components} und Systeme entworfen. Dabei werden in der Implementierung des Factory Spiel's alle Aspekte des datenorientierten Technolohgie-Stack's von Unity eingesetzt. Es werden die Daten in \textit{Components} gespeichert, welche von Systemen verarbeitet werden. Die Systeme werden größtenteils so geschrieben, dass Burst verwendet werden kann. Zusätzlich werden Job's erstellt, welche aus Systemen ausgeführt werden.\\
Auch hier zu Beginn das \textit{Component} für ein Item und das dazugehörige System:
\begin{lstlisting}[style=code, caption={Item Component ECS}, label=itemComponent]
public struct ItemComponent : IComponentData
{
    public int2 pos;
    public int itemID;
}
\end{lstlisting}
In dem \textit{Component} wird die Position und die ID des Items gespeichert. Was direkt auffällt ist die klare Trennung der Daten von der Logik, welche sich hier in einem System befindet. Das dazugehörige System sieht anders aus als das MonoBehaviour im Objektorientierten, die Logik ist jedoch dieselbe:
\begin{lstlisting}[style=code, caption={Item System und Job zum Bewegen von Items}]
[BurstCompile(CompileSynchronously = true)]
public partial struct ItemSystem : ISystem
{
    [BurstCompile(CompileSynchronously = true)]
    public void OnUpdate(ref SystemState state)
    {
    	//Job erstellen, Zeit übergeben und schedulen
        new ItemMoveJob
        {
            deltaTime = SystemAPI.Time.DeltaTime
        }.ScheduleParallel();
    }

    [BurstCompile(CompileSynchronously = true)]
    public partial struct ItemMoveJob : IJobEntity
    {
        public float deltaTime;
        
        [BurstCompile(CompileSynchronously = true)]
        private void Execute(ref LocalTransform transform, in ItemComponent item)
        {
        	//Gegenstand wird zu der übergebenen Position pos bewegt
            transform = transform.WithPosition(
                Vector3.Lerp(transform.Position.xyz,
                    new Vector3(item.pos.x, item.pos.y, -0.5f),
                deltaTime * 2f));
        }
    }
}
\end{lstlisting}
Das Item System ist durch das implementierte Interface \texttt{ISystem} als System defniniert und kann daher die \texttt{OnUpdate} Funktion nutzen. Bei dem Item System kommt zusätzlich auch der Burst Compiler und das Job System zum Einsatz. Das Attribut \texttt{BurstCompile} hilft dem Burst Compiler Methoden zu finden, welche mit Burst kompiliert werden sollen. \textit{CompileSynchronously} dient dem testen und besagt, dass erst das System durch den Burst Compiler kompiliert werden muss bevor es zum ersten Mal ausgeführt wird. Andernfalls könnte das System schon laufen, ohne den Burst Compiler genutzt zu haben. Die Methode \texttt{OnUpdate} wird ein mal pro Bild aufgerufen. Sie ist zu vergleichen mit der \texttt{Update} Methode in einem MonoBehaviour. Hier wird der \texttt{IJobEntity} ItemMoveJob erstellt. Dieser Job funktioniert mit einem LocalTransform Component, welches jedes Entity besitzt und die Position angibt, und mit dem ItemComponent aus \hyperref[itemComponent]{Listing \ref{itemComponent}}. Dabei wird das LocalTransform Component zum lesen und schreiben verwendet (erkennbar durch das Keywort ref) und das ItemComponent lediglich zum lesen (erkannbar durch das Keywort in). Auch hier wird, wie in dem Monobehaviour, nun die Position des Items mithilfe der Lerp Funktion von Vector3 und den Daten im ItemComponent verändert. Dieser Job, welcher die tatsächliche Logik für das Item enthält wird in der \texttt{OnUpdate} Methode parallel gescheduled. Dies ist hier speziell sehr vorteilhaft, da es sehr viele Items auf dem Spielfeld geben kann. Dadurch werden nicht tausende Items nacheinander bewegt, sondern alle parallel.\\
Die übergebene Position kommt, wie auch in dem Objektorientierten, von dem Förderband. Auch hier ist die Förderbandlogik in einem Job implementiert. Das hat den Vorteil, dass man dies wieder gut parallelisieren kann. Die Vorgehensweise, dass man die Items von hinten nach vorne ein Förderbandsegment weiterbewegt bleibt jedoch gleich. Ganz anders ist die Methode, wenn man Items erstellen will. Da man mit einem geschedulten Job sich nicht mehr auf dem \textit{Main Thread} befindet, kann man Items nicht mehr direkt erstellen.
\begin{lstlisting}[style=code, caption={Create Item System}]
[BurstCompile(CompileSynchronously = true)]
[UpdateAfter(typeof(ProcessingBuildingSystem))]
public partial struct CreateItemSystem : ISystem
{
	//Lookup um Daten eines anderen Entity auszulesen
    private ComponentLookup<InputConveyorComponent> inputLookup;

    [BurstCompile(CompileSynchronously = true)]
    public void OnCreate(ref SystemState state)
    {
      	//Für die OnUpdate Funktion wird das ItemEntitiesComponent gebraucht
      	//Darin sind alle Items für das Erstellen gespeichert
        state.RequireForUpdate<ItemEntitiesComponent>();
        inputLookup = state.GetComponentLookup<InputConveyorComponent>();
    }

    [BurstCompile(CompileSynchronously = true)]
    public void OnUpdate(ref SystemState state)
    {
        inputLookup.Update(ref state);
        var ecbSingleton = SystemAPI.GetSingleton<BeginSimulationEntityCommandBufferSystem.Singleton>();
        //Entity Command Buffer wird erstellt
        var ecb = ecbSingleton.CreateCommandBuffer(state.WorldUnmanaged);
        var itemEntities = SystemAPI.GetSingleton<ItemEntitiesComponent>();
        // Job wird erstellt und gescheduled
        new CreateItemJob
        {
            inputLookup = inputLookup,
            ecb = ecb,
            itemEntities = itemEntities
        }.Schedule();
        state.CompleteDependency();
    }
}
\end{lstlisting}
Hier sieht man eine Besonderheit des \textit{Entity Component Systems}, der \textit{Entity Command Buffer} (ECB). Dadurch dass strukturelle Änderungen nur auf dem \textit{Main Thread} passieren dürfen braucht man einen ECB um diese Änderungen zu sammeln und an späteren Stelle auf dem \textit{Main Thread} auszuführem. Hier wird der ECB verwendet um strukturelle Änderungen aus einem Job vorzunehmen.
\begin{lstlisting}[style=code, caption={Create Item Job}]
[BurstCompile(CompileSynchronously = true)]
[WithNone(typeof(OutputNotFoundTag))]
public partial struct CreateItemJob : IJobEntity
{
    public EntityCommandBuffer ecb;
    //Wenn nur von gelesen wird kann ReadOnly verwendet werden
    [ReadOnly] public ComponentLookup<InputConveyorComponent> inputLookup;
    [ReadOnly] public ItemEntitiesComponent itemEntities;

    [BurstCompile(CompileSynchronously = true)]
    //Job arbeitet mit dem Output eines Gebäudes
    private void Execute(ref OutputProcessingBuildingComponent output)
    {
        //Wenn der Output leer ist gibt es nichts zu tun
        if(output.itemID == -1) return;
        var itemID = output.itemID;
        var input = inputLookup[output.outputEntity];
        //Wenn Input des Förderbandes belegt ist wird auch keine
        //Änderung vorgenommen
        if(input.occupied || input.item != Entity.Null) return;
        //Item wird aus dem Output entfernt und mittels ECB
        //wird ein neues Item erstellt 
        output.itemID = -1;
        output.itemCreated = true;
        var itemEntity = itemEntities.GetEntityWithID(itemID);
        var item = ecb.Instantiate(itemEntity);
        //Position und Item Component des Items werden gesetzt
        ecb.SetComponent(item, LocalTransform.FromPositionRotationScale(
            new float3(output.pos.x, output.pos.y, -0.5f),
            quaternion.identity, 0.5f));
        ecb.SetComponent(item,
            new ItemComponent{pos = output.pos, itemID = itemID});
        //Input wird das Item zugewiesen
        ecb.SetComponent(output.outputEntity,
            new InputConveyorComponent{
            item = item, pos = input.pos, occupied = true});
    }
}
\end{lstlisting}
Der \textit{Entity Command Buffer} erstellt das Item und ändert noch in den \textit{Components} die Position des Items. Zusätzlich wird dem Input des Förderbandes das erstellte Item zugewiesen. Diese Änderungen werden in dem Job aufgenommen und nach dem Job in der richtigen Reihenfolge angewendet.