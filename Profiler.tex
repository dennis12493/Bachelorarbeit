\begin{figure}[H]
\centering
\begin{annotatedFigure}
	{\includegraphics[scale=0.428]{Bilder/Profiler.png}}
    \annotatedFigureBox{0.1688,0.9496}{0.363,1.01}{A}{0.363,0.9496}
	\annotatedFigureBox{0.12,0.65}{1.005,0.76}{B}{1.005,0.65}
	\annotatedFigureBox{-0.005,0.3778}{0.1259,1.01}{C}{0.1259,0.3778}
	\annotatedFigureBox{0.38,0.0888}{0.8705,0.3831}{D}{0.8705,0.0888}
\end{annotatedFigure}
\caption[Der \textit{Profiler} im Unity Editor]{Der \textit{Profiler} im Unity Editor. Mit ihm lassen sich die Auslastung des Spiels während des Spielens aufzeichnen und die Daten zur weiteren Verwendung abspeichern. In der oberen Leiste (A), lässt sich das Spiel wie gewünscht aufzeichnen und man sieht die Anzahl an aufgezeichneten Bildern. Man kann auch einzelne Bilder zur weiteren Inspektion auswählen. Darunter (B) wird der Verlauf der aufgezeichneten Bilder dargestellt. Hier sieht man für jedes Modul eine Auslastung über den Verlauf der 2000 Bilder. Links (C) werden die einzelnen Module angezeigt. Sichtbar sind hier die Module CPU Usage, Rendering und Memory, wobei man für weitere im Editor nach unten scrollen muss. Unten (D) ist eine genauere Timeline sichtbar, da hier nur einige Bilder abgebildet werden.}
\label{fig:profiler}
\end{figure}