\subsection{Job System} \label{jobs}
Das Job System von Unity ist ein leichter Weg um parallelen Code auszuführen. Jobs werden meist in der \texttt{OnUpdate} Funktion erzeugt und ausgeführt. Dabei kann man entscheiden, ob der Job auf dem \textit{Main Thread}, einem \textit{Worker Thread}, oder mehreren \textit{Worker Threads} ausgeführt werden soll. Die Anzahl an verfügbaren \textit{Worker Threads} wird dabei von der Anzahl an verfügbaren Kernen im Prozessor bestimmt. Nachdem der Job erstellt wurde, kann man ihn mit dem Funktionsaufruf \texttt{Run} (Ausführung auf dem \textit{Main Thread}), \texttt{Schedule} (Ausführung auf einem \textit{Worker Thread}), oder \texttt{ScheduleParallel} (Ausführung auf mehreren \textit{Worker Threads}) ausführen. Zusätzlich gibt es noch verschiedene Jobarten, welche auf die Arbeit mit dem \textit{ECS} angepasst wurden.
\subsubsection{Job mit einem \textit{Entity}}
Der Job mit einem \textit{Entity} ist der Standardjob, um über Daten von \textit{Components} zu iterieren. Er implementiert das Interface \texttt{IJobEntity}. Mit ihm lässt sich leicht definieren, welche Daten man lesen oder schreiben möchte. Zusätzlich kann man noch weitere Attribute, an die in dem Job befindlichen \texttt{Execute} Methode übergeben lassen. \hyperref[IJobEntity]{Listing \ref*{IJobEntity}} zeigt einen fertig implementierten Job, der das \texttt{IJobEntity} Interface implementiert.
\begin{lstlisting}[style=code, caption={[Beispiel für einen Job mit einem \textit{Entity} für eine einfache Addition]Beispiel für einen Job mit einem \textit{Entity} für eine einfache Addition. Der \texttt{Execute} Methode wird das \textit{Entity} und das \texttt{ExampleComponent} übergeben.}, label=IJobEntity]
//Beispiel Component
public struct ExampleComponent : IComponentData { public float Value; }
//Entities mit dem DontWantComponent werden ausgeschlossen
[WithNone(typeof(DontWantComponent))]
//Job ist eine Struktur und implementiert das IJobEntity Interface
public partial struct ExampleJob : IJobEntity
{
    //Execute Funktion wird für jedes ExampleComponent ausgeführt
    void Execute(Entity e, ref ExampleComponent component)
    {
    		//Addiert eins zu jedem ExampleComponent Wert.
        component.Value += 1f;
    }
}
//Beispiel System welches den Job verwendet
public partial class ExampleSystem : ISystem
{
    protected override void OnUpdate()
    {
        //Job wird erstellt und auf mehreren Worker Threads ausgeführt
        new ExampleJob().ScheduleParallel();
    }
}
\end{lstlisting}
Wie man in dem Beispiel sieht, ist der Job auch wieder eine Struktur und implementiert das \texttt{IJobEntity} Interface. Durch dieses Interface kann man eine eigene \texttt{Execute} Methode erstellen und anpassen. Je nach dem, welche \textit{Components} man der \texttt{Execute} Methode übergibt, werden alle \textit{Entities}, welche diese \textit{Components} besitzen, an die Funktion übergeben. Die übergebenen \textit{Components} sind dabei die des \textit{Entity}. Falls man die \textit{Entities} weiter einschränken oder lockern will, welche der Funktion übergeben werden, kann man dies mit den Attributen \texttt{WithNone(} \texttt{typeof(ComponentX)} \texttt{)} (\textit{Entities} mit dem ComponentX werden ausgeschlossen), \texttt{WithAll(} \texttt{typeof(ComponentX)} \texttt{)} (\textit{Entities} müssen das ComponentX besitzen), oder \texttt{WithAny(} \texttt{typeof(} \texttt{ComponentX)} \texttt{,} \texttt{typeof(ComponentY)} \texttt{)} (\textit{Entities} müssen das ComponentX, oder das ComponentY besitzen) über dem Job definieren.\\
In der \texttt{Execute} Funktion lässt sich dann mit den \textit{Components} arbeiten. \textit{Components}, welche man nur lesen will, sollte man mit dem Schlüsselwort \texttt{in} übergeben, \textit{Components}, welche man lesen und schreiben möchte, sollte man mit den Schlüsselwort \texttt{ref} (Zeile 9) übergeben. Dann kann man seine Logik so definieren, wie man sie braucht.
\subsubsection{Job mit einem \textit{Chunk}}
Der Job mit einem \textit{Chunk} implementiert das Interface \texttt{IJobChunk} und behandelt ganze \textit{Chunks} an Daten. Dem Job wird nicht nur ein einzelnes \textit{Entity} mit seinen \textit{Components} übergeben, sondern ein ganzer \textit{Chunk}, mit allen \textit{Components}, die darin enthalten sind. Er kann aus dem \textit{Chunk} das komplette Array eines \textit{Component} Typs holen und dieses dann in einer Schleife bearbeiten. Die Schleife geht die Anzahl an \textit{Entities} in dem \textit{Chunk} durch und indiziert das \textit{Component} Array. Der Job mit einem \textit{Chunk} ist komplexer zu implementieren, als ein Job mit einem \textit{Entity}. Er wird jedoch immer durch Codegenerierung von einem \texttt{IJobEntity} Job erzeugt, also ist in Wirklichkeit jeder Job mit einem \textit{Entity} eigentlich ein Job für den ganzen \textit{Chunk}. \hyperref[IJobChunk]{Listing \ref*{IJobChunk}} zeigt einen implementierten \texttt{IJobChunk} Job, welcher dieselbe Funktion hat, wie der \texttt{IJobEntity} Job in \hyperref[IJobEntity]{Listing \ref*{IJobEntity}}.
\begin{lstlisting}[style=code, caption={[Beispiel für einen Job mit einem \textit{Chunk} für eine einfache Addition]Beispiel für einen Job mit einem \textit{Chunk} für eine einfache Addition. Dieser ist analog zu dem Beispiel für ein Job mit einem \textit{Entity}.}, label=IJobChunk]
//Job der das IJobChunk Interface implementiert
public struct ExampleJob : IJobChunk
{
    //ComponentTypeHandle erlaubt es in der Execute Funktion auf die Components im Chunk zuzugreifen. Man braucht für jeden Component Typ einen eigenen ComponentTypeHandle
    public ComponentTypeHandle<ExampleComponent> exampleTypeHandle;

    //Execute Funktion, welche den kompletten Chunk übergeben bekommt. Der Integer unfilteredChunkIndex enthält den Index des Chunks welcher bearbeitet wird, da es, je nach Anzahl an Entities, auch mehr als einen geben kann. chunkEnabledMask ist eine Bitmaske. Falls das n'te Bit gesetzt ist, passt das n-te Entity in die Anforderungen des Jobs und sollte verarbeitet werden. Das Attribut useEnabledMask vereinfacht den Nutzen von chunkEnabledMask. Falls useEnabledMask false ist passen alle Entities zu den Anforderungen.
    public void Execute(in ArchetypeChunk chunk, int unfilteredChunkIndex,
        bool useEnabledMask, in v128 chunkEnabledMask)
    {
        //Man erhält ein Array aller Components von einem Chunk
        //durch den ComponentTypeHandle
        NativeArray<ExampleComponent> exampleComponents =
            chunk.GetNativeArray(ref exampleTypeHandle);
        //Der ChunkEntityEnumerator gibt mittels NextEntityIndex das nächste
        //zu bearbeitende Entity unter der Berücksichtigung von
        //chunkEnabledMask wieder.
        var enumerator = new ChunkEntityEnumerator(useEnabledMask,
            chunkEnabledMask, chunk.Count);
        //Schleife über alle Entities, die verarbeitet werden sollen
        while(enumerator.NextEntityIndex(out var i))
        {
            float3 newValue = exampleComponents[i].Value + 1;
            exampleComponents[i] = newValue;
        }
    }
}
\end{lstlisting}
Wie man sieht, ist der \texttt{IJobChunk} Job deutlich länger und komplexer als der \texttt{IJobEntity} Job, obwohl beide das Gleiche machen. Jedoch kann man viel besser erkennen, was wirklich bei der Ausführung des Jobs passiert. Die \textit{Chunks}, welche getrennt im Speicher liegen, werden der Execute Funktion übergeben. Welche das sind, kann man mit einer \textit{EntityQuery} bestimmen, welche man bei dem \texttt{IJobEntity} Job nicht unbedingt benötigt. Diese kann so aussehen:
\begin{lstlisting}[style=code, label=EntityQuery]
EntityQuery query = GetEntityQuery(typeof(ExampleComponent));
\end{lstlisting}
Mit dieser \textit{EntityQuery} kann man dann den Job ausführen:
\begin{lstlisting}[style=code, label=JobExecution]
protected override void OnUpdate(){
    var job = new ExampleJob();
    job.exampleTypeHandle = GetComponentTypeHandle<ExampleComponent>(false);
    //Dependency muss mit dem Job aktualisiert werden	
    this.Dependency = job.ScheduleParallel(query, this.Dependency);
}
\end{lstlisting}
Zusätzlich braucht man, da man \textit{Components} auch deaktivieren kann, eine Bitmaske, welche verhindert, dass \textit{Entities} bearbeitet werden, deren \textit{Components} deaktiviert sind. Ein \texttt{IJobEntity} generiert den \texttt{IJobChunk} so, dass automatisch darauf geachtet wird. Deshalb wird empfohlen den Job mit einem \textit{Entity} zu nutzen, welcher einfacher zu implementieren ist.