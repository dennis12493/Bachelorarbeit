\section{Fazit}
datenorientierte Programmierung hat großes Potenzial in der Spieleentwicklung. Wie man in der Auswertung gesehen hat ist das \textit{Entity Component System} dem Mono sehr überlegen. Gerade dann, wenn es sehr viele Objekte in der Spielwelt gibt, funktioniert die Verarbeitung der Daten mit einem datenorientierten Ansatz wesentlich schneller. Mit dem Zusammenspiel von Daten und System, welche auf die reine Verarbeitung der Daten spezialisiert sind, kann man außerdem viel Parallelisieren. Dies führt zu einer hohen Effizienz, da der Prozessor besser genutzt wird. Dazu kommt, dass modernere CPU's mehr Kerne haben und diese mit Parallelisierung besser genutzt werden.\\
Jedoch muss man auch betonen, dass das gewählte Beispiel Vorteilhaft für die datenorientierte Programmierung war. Es kommt also auch auf die Anwendung an, ob ein datenorientierter Ansatz einen großen Vorteil bietet. Hat man viel UI, oder ein insgesamt eher kleines Spiel, bringt ein datenorientierter Ansatz nicht viel Performance Optimierung. Auch ist das \textit{Entity Component System} von Unity aufwändig in der Programmierung. Es gibt viel Overhead, vor allem wenn man Objekt zur Laufzeit erstellen möchte. Hier ist jedoch Unity noch in der Entwicklung und wird in Zukunft die Umsetzung des \textit{ECS} einfacher gestalten.\\
Kommt vor allem auf die Anwendung an. Viele gleiche Objekte prima für datenorientierte Programmierung, da Verarbeitung dann sehr schnell passieren kann. Parallelisierung von Code ist immer gut, neuere CPU's haben immer mehr Kerne, jedoch auch komplexer und man hat zusätzliche Schwierigkeiten. Unity hat momentan noch sehr viel Overhead was die Codegröße betrifft, wird sich mit der Zeit aber auch noch ändern. Schwierig zu entscheiden wird es bei Projekten mit sehr viel UI / wenig gleichen Objekten.