\section{Fazit}
Bis jetzt nur Stichpunkte:\\
Datenorientierte Programmierung hat großes Potenzial in der Spieleentwicklung. Kommt vor allem auf die Anwendung an. Viele gleiche Objekte prima für Datenorientierte Programmierung, da Verarbeitung dann sehr schnell passieren kann. Parallelisierung von Code ist immer gut, neuere CPU's haben immer mehr Kerne, jedoch auch komplexer und man hat zusätzliche Schwierigkeiten. Unity hat momentan noch sehr viel Overhead was die Codegröße betrifft, wird sich mit der Zeit aber auch noch ändern. Schwierig zu entscheiden wird es bei Projekten mit sehr viel UI / wenig gleichen Objekten.