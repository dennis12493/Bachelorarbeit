\section{Fazit}
Zusammenfassend lässt sich sagen, dass die Datenorientierte Programmierung großes Potenzial in der Spieleentwicklung hat, insbesondere bei Videospielen mit vielen Objekten in der Spielwelt. Das \textit{Entity Component System} hat sich in der Auswertung als deutlich überlegen gegenüber dem Monobehaviour herausgestellt. Insbesondere bietet die Cache Freundlichkeit des datenorientierten Ansatzes eine Möglichkeit, die Daten zu verarbeiten, sodass weniger Cache-Misses entstehen. Dadurch wird eine schnellere Verarbeitung ermöglicht. Durch das Zusammenspiel von Daten und Systemen wird zudem eine effiziente Parallelisierung ermöglicht, wodurch Prozessoren mit mehreren Kernen effizienter genutzt werden können.\\
Es ist jedoch wichtig zu betonen, dass die Vorteile der Datenorientierten Programmierung stark von der spezifischen Anwendung abhängen. Bei Spielen mit insgesamt wenig Objekten bietet ein datenorientierter Ansatz möglicherweise nicht mehr Performance.\\
Auch ist die Implementierung des \textit{Entity Component Systems} von Unity derzeit noch aufwändig und mit einem gewissen Overhead bei der Programmierung verbunden, insbesondere bei der Erstellung von Objekten zur Laufzeit. Es ist jedoch zu erwarten, dass Unity in Zukunft die Umsetzung des \textit{ECS} vereinfachen wird.\\
Die Nutzung von Unity's Jobs mit mehreren Threads bringt ebenfalls einen Overhead mit sich. Dadurch muss man abwägen, ab wann sich Jobs, insbesondere parallelisierte Jobs, lohnen. Bei einer hohen Anzahl an \textit{Entities} mit wenig Abhängigkeiten ist das Job System sehr vorteilhaft, auch weil es sehr einfach umzusetzen ist.\\
Die Nutzung des Burst Compilers ist immer vorteilhaft, jedoch ist er nur mit unverwalteten Daten und einem beschränkten C\# Set kompatibel. Dadurch ist es schwierig so viele Funktionalitäten wie möglich kompatibel zu machen. Wenn jedoch eine Funktionalität mit Burst kompatibel ist, bietet der Burst Compiler nur Vorteile.\\
Insgesamt kann Datenorientierte Programmierung in der Spieleentwicklung eine leistungsstarke Alternative zu der Objektorientierten Programmierung sein. Wie man in der Auswertung gesehen hat, kann bei geeigneten Anwendungen eine erheblich erhöhte Performance möglich sein.