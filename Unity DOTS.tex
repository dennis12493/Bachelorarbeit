\section[Unity's Datenorientierter Technologie-Stack]{Unity's Datenorientierter Technologie-Stack\protect{\footnote{\url{https://unity.com/de/dots}}}}
Die vorliegende Arbeit basiert auf dem \textit{com.unity.entities} Package mit der Version 1.0.0-pre.65\footnote{\url{https://docs.unity3d.com/Packages/com.unity.entities@1.0/changelog/CHANGELOG.html}}. Dies ist, Stand dem 01.04.2023, die aktuellste Version von Unity's \textit{Enitity Component System}. Da die Version nicht final fertiggestellt wurde und sich noch im Entwicklungsstadium befindet, kann sich mit der Zeit viel an der Art und Weise ändern, wie Unity Datenorientierte Programmierung umsetzt. Das Grundkonzept der Datenorientierung Programmierung ist allerdings mit dem \textit{Datenorientierten Technologie-Stack} festgeschrieben.\\
Unity's datenorientierter Ansatz wird durch Unity's \textit{Datenorientierten Technologie-Stack} umgesetzt. Dieser besteht aus drei Teilen:\\
1. Das \textit{Entity Component System} (\textit{ECS}) für Unity. Dies ist ein datenorientiertes Framework für Unity. Mit dem \textit{ECS} lässt sich der datenorientierte Ansatz umsetzen.\\
3. Das C\# Job System. Das Job System von Unity erlaubt es parallelen Code zu schreiben, welcher sicher und schnell läuft.\\
2. Der Burst Compiler. Der Burst Compiler übersetzt von IL Code zu optimiertem nativem Code. Er nutzt die LLVM Compiler Infrastruktur.\\
Diese drei Teile werden in den folgenden Katpiteln \ref{ecs}, \ref{jobs} und \ref{burst} erklärt.
\subsection{Objektorientierte Programmierung in Unity}
Bevor das \textit{Entity Component System} betrachtet wird, ist ein kleiner Exkurs zu der herkömm\-lichen Weise, wie in Unity gearbeitet wird, sinnvoll. Dies ist hilfreich um die folgenden Kapitel besser zu verstehen. In Unity basiert alles auf \textit{GameObjects}, die das Objektorientierte schon im Namen haben. Die \textit{GameObjects} werden immer in einer Szene erstellt, wobei es mehrere Szenen in einem Projekt geben kann. Es kann beispielsweise eine Szene für den Startbildschirm geben und eine Szene für das eigentliche Spiel. Alles was man in einer Szene erstellt, ist zunächst ein \textit{GameObject} welches man dann mit Funktionalitäten füllt. Durch Komponenten, die man dem \textit{GameObject} gibt, kann man sie zu Licht, Charakteren oder Gegenständen werden lassen. Die Komponenten sind nicht zu verwechseln mit den \textit{Components} aus dem \textit{Entity Component System}. Diese werden in \hyperref[components]{Kapitel \ref{components}} erläutert. In den folgenden Abschnitten erkennt man hierbei jedoch viele Parallelen zu dieser Arbeitsweise. Die Art wie man Code schreibt ist aber unterschiedlich. Zugriffe auf Daten anderer \textit{GameObjects} funktionieren, indem man erst auf das \textit{GameObject} zugreift und dann darüber auf seine Komponenten. Also ein typischer objektorientierter Ansatz.
\subsection{Das \textit{Entity Component System}} \label{ecs}
Das \textit{Entity Component System} besteht aus \textit{Entities}, \textit{Components} und Systemen. \textit{Entities} zeigen auf, welche \textit{Components} zusammen gehören und Systeme verarbeiten die \textit{Components}. Sie sind also für das Transformieren der Daten zuständig. \hyperref[fig:ecs_concept]{Abbildung \ref*{fig:ecs_concept}} zeigt das Konzept des \textit{Entity Component Systems}.
\begin{figure}[H]
\begin{center}
\includegraphics[scale=0.8]{Bilder/ECSConcept.png}
\caption[Zusammenspiel von \textit{Entities}, \textit{Components} und Systemen]{Zusammenspiel von \textit{Entities}, \textit{Components} und Systemen. Beispielhaft werden hier drei \textit{Entities} gezeigt, wobei \textit{Entity} A und B vier \textit{Components} haben und \textit{Entity} C nur drei \textit{Components} hat. \textit{Entity} C fehlt das \texttt{Renderer} \textit{Component}. Das System nimmt als Eingabe alle \texttt{Translation} und \texttt{Rotation} \textit{Components} und modifiziert damit das \texttt{LocalToWorld} \textit{Component} der \textit{Entities}.\\
\footnotesize{Quelle: \url{https://docs.unity3d.com/Packages/com.unity.entities@0.50/manual/ecs_core.html}}}
\label{fig:ecs_concept}
\end{center}
\end{figure}
\subsubsection[\textit{Entities}]{\textit{Entities}\protect{\footnote{\url{https://docs.unity3d.com/Packages/com.unity.entities@1.0/manual/concepts-entities.html}}}}\textit{Entities} repräsentieren meist Dinge in einem Unity Spiel. Das können der Spielcharakter, Gegenstände, oder beliebige Gegner sein. Sie können aber auch abstrakte Dinge, wie beispielsweise Events repräsentieren. Ein \textit{Entity} ist vergleichbar mit einem \textit{GameObject} im objektorientierten Ansatz von Unity. \textit{Entities} besitzen hierbei jedoch weder Daten noch ein Verhalten, sondern zeigen lediglich auf, welche Daten beziehungsweise \textit{Components} zueinander gehören. Alle \textit{Entities} in einer Spielwelt gehören zu einer sogenannten Welt\footnote{\url{https://docs.unity3d.com/Packages/com.unity.entities@0.1/manual/world.html}}. Zu dieser Welt gehört genau ein \textit{EntityManager}\footnote{\url{https://docs.unity3d.com/Packages/com.unity.entities@1.0/api/Unity.Entities.EntityManager.html}}. Der \textit{EntityManager} organisiert alle \textit{Entities} in dieser Welt. Mit ihm lassen sich \textit{Entities} erstellen, zerstören, \textit{Components} zu \textit{Entities} hinzufügen, entfernen oder verändern.
\subsubsection[\textit{Components}]{\textit{Components}\protect{\footnote{\url{https://docs.unity3d.com/Packages/com.unity.entities@1.0/manual/concepts-components.html}}}} \label{components}
\textit{Components} speichern die Daten eines \textit{Entity}. Diese Daten werden von Systemen genutzt und verarbeitet. Dabei unterscheidet man zwischen verwalteten \textit{Components} und unverwalteten \textit{Components}. Unverwaltete \textit{Components} werden in Unity als C\# Strukturen implementiert, welche leichtgewichtiger als Klassen sind. Diese Strukturen können auch nur unverwaltete Daten speichern. Unverwaltete Daten sind beispielsweise Integer, Boolean, Bytes, Chars, oder andere Strukturen. Verwaltete \textit{Components} werden hingegen als Klassen definiert und können alle Daten halten. Es ist jedoch üblich unverwaltete \textit{Components} zu verwenden, da diese nicht so resourcenintensiv im Speichern und Zugreifen sind. Um ein unverwaltetes Component zu erstellen, kann man das \texttt{IComponentData} Interface verwenden. \hyperref[lstExampleComponent]{Listing \ref*{lstExampleComponent}} zeigt ein unverwaltetes \textit{Component}.
\begin{lstlisting}[style=code, caption={[Beispiel eines unverwalteten \textit{Components}]Beispiel eines unverwalteten \textit{Components}. Das \textit{Component} speichert die Position und die Geschwindigkeit eines \textit{Entity}.}, label=lstExampleComponent]
public struct ExampleComponent : IComponentData
{
    public int2 position;
    public float speed;
}
\end{lstlisting}
Es gibt aber auch andere Arten von \textit{Components}. Es gibt \textit{Shared Components}, \textit{Buffer Components}, etc\footnote{\url{https://docs.unity3d.com/Packages/com.unity.entities@1.0/manual/components-type.html}}. Es sind hier nicht alle \textit{Components} relevant, da manche lediglich die Spieleentwicklung mit dem datenorientierten Ansatz erleichtern, aber keine neue Funktion bieten. Zudem gibt es alle Arten von \textit{Components} sowohl als verwaltete als auch unverwaltete \textit{Components}.\\
\textbf{\textit{Shared Components}}: \textit{Shared Components} sind, wie der Name schon sagt, unter den \textit{Entities} geteilt. Sie gruppieren die \textit{Entities} nach dem Wert des \textit{Shared Component}. \textit{Shared Components} werden abseits anderer \textit{Components} gespeichert und sind ein Weg um Datenduplizierung zu vermeiden. Unity speichert zusätzlich alle \textit{Entities}, welche die gleiche Kombination aus \textit{Component} Typen haben und das gleiche \textit{Shared Component} besitzen, also auch den gleichen Wert, gemeinsam. Dies hat zwar Vorteile, aber auch den großen Nachteil, dass das Ändern von Werten des \textit{Shared Component} ein Verschieben der \textit{Entities} im Speicher zufolge hat. Die Probleme hierbei findet man in Kapitel \ref{structuralChanges}. Einen sehr großen Vorteil haben \textit{Shared Components} in jedem Spiel das mit Unity's \textit{ECS} entwickelt wird. Das \texttt{RenderMesh} \textit{Component}, also das \textit{Component} welches für das Aussehen der \textit{Entities} zuständig ist, ist immer ein \textit{Shared Component}. Dies ist sinnvoll, da sich dieses \textit{Component} sehr selten im Wert ändert und viele \textit{Entities} das selbe \texttt{RenderMesh} \textit{Component} besitzen. Das kann ganz simpel einfach das Aussehen eines Baumes in einem Spiel sein. Oft gibt es viele Bäume in dem Spiel und diese ändern auch ihr Aussehen nicht.\\
\textbf{\textit{Buffer Components}}: Falls man mehrere \textit{Components} der selben Art für ein \textit{Entity} braucht, sind \textit{Buffer Components} sehr hilfreich. Diese agieren wie ein Array von \textit{Components}. Das ist besonder hilfreich, wenn man einem \textit{Entity} ein Inventar geben möchte, da dieses aus mehreren Gegenständen bestehen kann. \hyperref[lstBufferExample]{Listing \ref*{lstBufferExample}} zeigt, wie man ein Inventar entwerfen würde.
\begin{lstlisting}[style=code, caption={[Beispiel eines \textit{Buffer Components}]Beispiel eines \textit{Buffer Components}. Das \textit{Buffer Component} stellt ein Inventar dar, welches verschiedene Items mit ihrer Anzahl speichert. Das Array ist maximal acht Elemente groß.}, label=lstBufferExample]
//Array Kapazität auf acht festlegen
[InternalBufferCapacity(8)]
public struct ItemAmount : IBufferElementData
{
    public int itemID;
    public int amount;
}
\end{lstlisting}
Wie man sieht, leitet die Struktur diesmal von \texttt{IBufferElementData} ab, welche den Typ \textit{Buffer Component} angibt. Zusätzlich wird hier das Attribut \texttt{InternalBufferCapacity} genutzt. Damit legt man die Kapazität des \textit{Components} fest. Standardmäßig ist die Kapazität die Anzahl an Elementen, welche in 128 Bytes passen. Also hätten wir bei diesem Beispiel eine Kapazität von 16, da in einem \textit{Component} zwei Integer à 32 bit gespeichert werden. Jedoch braucht man für das Inventar beispielsweise nur 8 zu speichernde Gegenstände und kann so den benötigten Speicherplatz reduzieren.
\subsubsection[Systeme]{Systeme\protect{\footnote{\url{https://docs.unity3d.com/Packages/com.unity.entities@1.0/manual/concepts-systems.html}}}}
Systeme beschreiben das Verhalten und beinhalten die Logik zum Transformieren der Daten. Die Systeme werden ein Mal pro ausgegebenem Bild mithilfe der \texttt{OnUpdate} Funktion auf dem \textit{Main Thread} ausgeführt. Genau wie im objektorientierten Ansatz von Unity gibt es auch hier mehrere Methoden, die zum Start oder Ende ausgeführt werden. Zusätzlich kann man unter den definierten Systemen eine Reihenfolge festlegen, in der diese ausgeführt werden sollen. So wie bei den \textit{Components} gibt es auch bei den Systemen eine Klasse für verwaltete Daten (welche in diesem Fall von der Klasse \texttt{SystemBase} erbt) und eine Struktur für unverwaltete Daten (welche in diesem Fall das Interface \texttt{ISystem} implementiert). Zusätzlich sind Systeme immer an eine Welt gebunden. \hyperref[lstSystemExample]{Listing \ref*{lstSystemExample}} zeigt, wie ein System für unverwaltete Daten und ohne implementierte Logik aussieht.
\begin{lstlisting}[style=code, caption={[Beispiel eines Systems]Beispiel eines Systems. Das System hat verschiedene Methoden, welche zum Start beziehungsweise Ende ausgeführt werden und eine \texttt{OnUpdate} Methode. Diese wird einmal pro Bild ausgeführt.}, label=lstSystemExample]
public partial struct ExampleSystem : ISystem, ISystemStartStop
{
    //Wird beim Erstellen des Systems ausgeführt
    public void OnCreate(ref SystemState state){}
    //Wird vor dem ersten OnUpdate des Systems ausgeführt
    public void OnStartRunning(ref SystemState state){}
    //Wird für jedes ausgegebene Bild ausgeführt
    public void OnUpdate(ref SystemState state){}
    //Wird beim Stoppen des Systems ausgeführt
    public void OnStopRunning(ref SystemState state){}
    //Wird beim Zerstören des Systems ausgeführt
    public void OnDestroy(ref SystemState state){}
}
\end{lstlisting}
Dabei ist das Interface \texttt{ISystemStartStop} optional und bietet die Möglichkeit beim Starten und Stoppen des Systems zusätzliche Logik auszuführen. Wie man sieht, wird allen Methoden auch eine Referenz des \texttt{SystemState} übergeben. Darüber kann man auf verschiedene nützliche Dinge zugreifen, wie beispielsweise die Welt, den \textit{EntityManager}, oder aber auch alle \textit{Components} eines Typs. Eine sinnvolle Herangehensweise ist es, in der \texttt{OnUpdate} Methode Jobs zu schedulen. Dies wird in Kapitel \ref{jobs} weiter beschrieben.
\subsubsection[Archetypen]{Archetypen\protect{\footnote{\url{https://docs.unity3d.com/Packages/com.unity.entities@1.0/manual/concepts-archetypes.html}}}}
Ein Archetyp ist eine gewisse Zusammenstellung aus \textit{Components}. Jedes \textit{Entity} kann somit einem Archetyp zugeordnet werden. Beispielsweise sind alle \textit{Entities}, welche nur das \texttt{Translation} \textit{Component} (Position) haben, einem Archetyp zugeordnet. \textit{Entities}, welche zusätzlich \textit{Component} A besitzen, gehören zu einem anderen Archetyp. \hyperref[fig:archetype_concept]{Abbildung \ref*{fig:archetype_concept}} zeigt das Konzept von Archetypen.
\begin{figure}[H]
\begin{center}
\includegraphics[scale=0.66]{Bilder/ArchetypeConcept.png}
\caption[Konzept von Archetypen]{Konzept von Archetypen. \textit{Entities}, welche die gleiche Zusammenstellung von \textit{Components} haben gehören einem Archetyp an. \textit{Entity} A und B gehören durch die gleiche Zusammenstellung an \textit{Components} also dem Archetyp M an. \textit{Entity} C hat eine andere Zusammenstellung und gehört Archetyp N an.\\
\footnotesize{Quelle: \url{https://docs.unity3d.com/Packages/com.unity.entities@0.50/manual/ecs_core.html}}}
\label{fig:archetype_concept}
\end{center}
\end{figure}
Durch Archetypen ist es möglich, sehr performant, datenorientiert zu arbeiten. Möchte man in einem System auf verschiedenen \textit{Components} Operationen ausführen, kann man alle Archetypen nach diesen \textit{Components} durchsuchen und muss nicht alle \textit{Entities} durchsuchen. Zusätzlich kann man diese Anfragen an Archetypen cachen um noch mehr Performance zu erreichen. Unity speichert alle \textit{Components} von \textit{Entities} für einen gewissen Archetyp in einem Block. Dieser wird auch \textit{Chunk} genannt. Abbildung \ref{fig:archetyp_chunks} zeigt, wie Archetypen mit \textit{Chunks} zusammenhängen.
\begin{figure}[H]
\begin{center}
\includegraphics[scale=0.45]{Bilder/ArchetypeChunkDiagram.png}
\caption[Konzept von \textit{Chunks}]{Konzept von \textit{Chunks}. \textit{Chunks} sind Speicherbereiche für Archetypen. In dem Bild ist zu erkennen, dass der Archetyp links vier \textit{Components} hat (Anzahl an Reihen), der Archetyp in der Mitte drei \textit{Components} und der Archetyp rechts zwei \textit{Components}. In einen \textit{Chunk} passen hier jeweils vier \textit{Entities}, wobei dies von der Anzahl und Größe der jeweiligen \textit{Components} abhängig ist. Wenn ein \textit{Chunk} voll ist, muss ein neuer \textit{Chunk} erstellt werden. Daher kann es, abhängig von der Anzahl der \textit{Entities} in einem \textit{Chunk}, unterschiedlich viele \textit{Chunks} pro Archetyp geben.\\
\footnotesize{Quelle: \url{https://docs.unity3d.com/Packages/com.unity.entities@0.50/manual/ecs_core.html}}}
\label{fig:archetyp_chunks}
\end{center}
\end{figure}
Jeder dieser \textit{Chunks} ist 16 KiB groß. Demnach hängt es von den \textit{Components} ab, wie viele \textit{Entities} in einen \textit{Chunk} passen. Der \textit{Chunk} beinhaltet ein Array für jeden Typ der \textit{Components} und zusätzlich ein Array für die ID's der \textit{Entities}. Pro Arrayindex wird je ein \textit{Entity} gespeichert. Im Index 0 aller Arrays werden die Daten des ersten \textit{Entities} gespeichert. Falls ein \textit{Entity} zerstört, oder in einen anderen \textit{Chunk} bewegt wird (falls ein \textit{Component} hinzugefügt oder entfernt wird), wird das letzte \textit{Entity} an seine Stelle bewegt. Falls ein \textit{Chunk} voll ist, erstellt der \textit{EntityManager} einen neuen \textit{Chunk}, wenn ein \textit{Entity} hinzukommt. Leere \textit{Chunks} werden gelöscht.
\subsubsection{Strukturelle Änderungen}\label{structuralChanges}
Strukturelle Änderungen sind eines der wenigen unperformantem Operationen in Unity's \textit{ECS}. Strukturelle Änderungen können das Erstellen und Zerstören eines \textit{Entities}, das Hinzufügen und Entfernen von \textit{Components} oder das Ändern von Daten eines \textit{Shared Components} sein\footnote{https://docs.unity3d.com/Packages/com.unity.entities@1.0/manual/concepts-structural-changes.html}. Also im Grunde Operationen, welche das Ändern eines, oder mehrerer \textit{Chunks} erfordern. Solche Änderungen könnten andere zur selben Zeit ausgeführte Aktionen invalidieren und müssen deshalb auf dem \textit{Main Thread} ausgeführt werden. Um dennoch Änderungen dynamisch an beliebiger Stelle ausführen zu können, nutzt man den \textit{Entity Command Buffer} (\textit{ECB}). Mit dem \textit{ECB} lassen sich strukturelle Änderungen sammeln und zu einem späteren Zeitpunkt in einer festgelegt Reihenfolge ausführen. So lassen sich problemlos aus einem Job strukturelle Änderungen sammeln und nach Beendigung des Jobs, diese auf dem \textit{Main Thread} ausführen. \hyperref[lstECBExample]{Listing \ref*{lstECBExample}} zeigt, wie ein \textit{ECB} funktioniert.
\begin{lstlisting}[style=code, caption={[Beispiel eines \textit{Entity Command Buffers}]Beispiel eines \textit{Entity Command Buffers}. Der \textit{ECB} wird erstellt, es werden strukturelle Änderungen vorgenommen und diese werden auf dem \textit{Main Thread} abgespielt.}, label=lstECBExample]
public void OnUpdate(ref SystemState state)
{
    //Neuer ECB wird erstellt
    EntityCommandBuffer ecb = new EntityCommandBuffer(Allocator.TempJob);
    //Job wird erstellt und der ECB wird übergeben
    new ExampleJob
    {
        ecb = ecb
    //Mit Schedule wird der Job gestartet
    }.Schedule();
    state.CompleteDependency();
    //Strukturelle Änderungen werden abgespielt
    ecb.Playback(state.EntityManager);
    //ECB muss auch wieder disposed werden
    ecb.Dispose();
}
\end{lstlisting}
Man erstellt einen \textit{ECB}, reiht verschiedene Aktionen in die Schlange ein und spielt diese auf dem \textit{Main Thread} wieder ab. Danach sollte man den \textit{ECB} wieder disposen. Das gibt den Speicher für das Objekt wieder frei und räumt gegebenenfalls weitere Ressourcen auf. In dem Job kann man mit dem übergebenen \textit{ECB} verschiedene Aktionen durchführen. Diese können beispielsweise so aussehen:
\begin{lstlisting}[style=code]
//Ein Entity erstellen:
Entity newEntity = ecb.Instantiate(e);
//Dem Entity ein Component hinzufügen:
ecb.AddComponent<ExampleComponent>(newEntity);
\end{lstlisting}
Wie ein Job genau aussieht und funktioniert wird in Kapitel \ref{jobs} beschrieben.
\subsection{Job System} \label{jobs}
Das Job System von Unity ist ein leichter Weg um parallelen Code auszuführen. Jobs werden meist in der \texttt{OnUpdate} Funktion erzeugt und ausgeführt. Dabei kann man entscheiden, ob der Job auf dem \textit{Main Thread}, einem \textit{Worker Thread}, oder mehreren \textit{Worker Threads} ausgeführt werden soll. Die Anzahl an verfügbaren \textit{Worker Threads} wird dabei von der Anzahl an verfügbaren Kernen im Prozessor bestimmt. Nachdem der Job erstellt wurde, kann man ihn mit dem Funktionsaufruf \texttt{Run} (Ausführung auf dem \textit{Main Thread}), \texttt{Schedule} (Ausführung auf einem \textit{Worker Thread}), oder \texttt{ScheduleParallel} (Ausführung auf mehreren \textit{Worker Threads}) ausführen. Zusätzlich gibt es noch verschiedene Jobarten, welche auf die Arbeit mit dem \textit{ECS} angepasst wurden.
\subsubsection{Job mit einem \textit{Entity}}
Der Job mit einem \textit{Entity} ist der Standardjob, um über Daten von \textit{Components} zu iterieren. Er implementiert das Interface \texttt{IJobEntity}. Mit ihm lässt sich leicht definieren, welche Daten man lesen oder schreiben möchte. Zusätzlich kann man noch weitere Attribute, an die in dem Job befindlichen \texttt{Execute} Methode übergeben lassen. \hyperref[IJobEntity]{Listing \ref*{IJobEntity}} zeigt einen fertig implementierten Job, der das \texttt{IJobEntity} Interface implementiert.
\begin{lstlisting}[style=code, caption={[Beispiel für einen Job mit einem \textit{Entity} für eine einfache Addition]Beispiel für einen Job mit einem \textit{Entity} für eine einfache Addition. Der \texttt{Execute} Methode wird das \textit{Entity} und das \texttt{ExampleComponent} übergeben.}, label=IJobEntity]
//Beispiel Component
public struct ExampleComponent : IComponentData { public float Value; }
//Entities mit dem DontWantComponent werden ausgeschlossen
[WithNone(typeof(DontWantComponent))]
//Job ist eine Struktur und implementiert das IJobEntity Interface
public partial struct ExampleJob : IJobEntity
{
    //Execute Funktion wird für jedes ExampleComponent ausgeführt
    void Execute(Entity e, ref ExampleComponent component)
    {
    		//Addiert eins zu jedem ExampleComponent Wert.
        component.Value += 1f;
    }
}
//Beispiel System welches den Job verwendet
public partial class ExampleSystem : ISystem
{
    protected override void OnUpdate()
    {
        //Job wird erstellt und auf mehreren Worker Threads ausgeführt
        new ExampleJob().ScheduleParallel();
    }
}
\end{lstlisting}
Wie man in dem Beispiel sieht, ist der Job auch wieder eine Struktur und implementiert das \texttt{IJobEntity} Interface. Durch dieses Interface kann man eine eigene \texttt{Execute} Methode erstellen und anpassen. Je nach dem, welche \textit{Components} man der \texttt{Execute} Methode übergibt, werden alle \textit{Entities}, welche diese \textit{Components} besitzen, an die Funktion übergeben. Die übergebenen \textit{Components} sind dabei die des \textit{Entity}. Falls man die \textit{Entities} weiter einschränken oder lockern will, welche der Funktion übergeben werden, kann man dies mit den Attributen \texttt{WithNone(} \texttt{typeof(ComponentX)} \texttt{)} (\textit{Entities} mit dem ComponentX werden ausgeschlossen), \texttt{WithAll(} \texttt{typeof(ComponentX)} \texttt{)} (\textit{Entities} müssen das ComponentX besitzen), oder \texttt{WithAny(} \texttt{typeof(} \texttt{ComponentX)} \texttt{,} \texttt{typeof(ComponentY)} \texttt{)} (\textit{Entities} müssen das ComponentX, oder das ComponentY besitzen) über dem Job definieren.\\
In der \texttt{Execute} Funktion lässt sich dann mit den \textit{Components} arbeiten. \textit{Components}, welche man nur lesen will, sollte man mit dem Schlüsselwort \texttt{in} übergeben, \textit{Components}, welche man lesen und schreiben möchte, sollte man mit den Schlüsselwort \texttt{ref} (Zeile 9) übergeben. Dann kann man seine Logik so definieren, wie man sie braucht.
\subsubsection{Job mit einem \textit{Chunk}}
Der Job mit einem \textit{Chunk} implementiert das Interface \texttt{IJobChunk} und behandelt ganze \textit{Chunks} an Daten. Dem Job wird nicht nur ein einzelnes \textit{Entity} mit seinen \textit{Components} übergeben, sondern ein ganzer \textit{Chunk}, mit allen \textit{Components}, die darin enthalten sind. Er kann aus dem \textit{Chunk} das komplette Array eines \textit{Component} Typs holen und dieses dann in einer Schleife bearbeiten. Die Schleife geht die Anzahl an \textit{Entities} in dem \textit{Chunk} durch und indiziert das \textit{Component} Array. Der Job mit einem \textit{Chunk} ist komplexer zu implementieren, als ein Job mit einem \textit{Entity}. Er wird jedoch immer durch Codegenerierung von einem \texttt{IJobEntity} Job erzeugt, also ist in Wirklichkeit jeder Job mit einem \textit{Entity} eigentlich ein Job für den ganzen \textit{Chunk}. \hyperref[IJobChunk]{Listing \ref*{IJobChunk}} zeigt einen implementierten \texttt{IJobChunk} Job, welcher dieselbe Funktion hat, wie der \texttt{IJobEntity} Job in \hyperref[IJobEntity]{Listing \ref*{IJobEntity}}.
\begin{lstlisting}[style=code, caption={[Beispiel für einen Job mit einem \textit{Chunk} für eine einfache Addition]Beispiel für einen Job mit einem \textit{Chunk} für eine einfache Addition. Dieser ist analog zu dem Beispiel für ein Job mit einem \textit{Entity}.}, label=IJobChunk]
//Job der das IJobChunk Interface implementiert
public struct ExampleJob : IJobChunk
{
    //ComponentTypeHandle erlaubt es in der Execute Funktion auf die Components im Chunk zuzugreifen. Man braucht für jeden Component Typ einen eigenen ComponentTypeHandle
    public ComponentTypeHandle<ExampleComponent> exampleTypeHandle;

    //Execute Funktion, welche den kompletten Chunk übergeben bekommt. Der Integer unfilteredChunkIndex enthält den Index des Chunks welcher bearbeitet wird, da es, je nach Anzahl an Entities, auch mehr als einen geben kann. chunkEnabledMask ist eine Bitmaske. Falls das n'te Bit gesetzt ist, passt das n-te Entity in die Anforderungen des Jobs und sollte verarbeitet werden. Das Attribut useEnabledMask vereinfacht den Nutzen von chunkEnabledMask. Falls useEnabledMask false ist passen alle Entities zu den Anforderungen.
    public void Execute(in ArchetypeChunk chunk, int unfilteredChunkIndex,
        bool useEnabledMask, in v128 chunkEnabledMask)
    {
        //Man erhält ein Array aller Components von einem Chunk
        //durch den ComponentTypeHandle
        NativeArray<ExampleComponent> exampleComponents =
            chunk.GetNativeArray(ref exampleTypeHandle);
        //Der ChunkEntityEnumerator gibt mittels NextEntityIndex das nächste
        //zu bearbeitende Entity unter der Berücksichtigung von
        //chunkEnabledMask wieder.
        var enumerator = new ChunkEntityEnumerator(useEnabledMask,
            chunkEnabledMask, chunk.Count);
        //Schleife über alle Entities, die verarbeitet werden sollen
        while(enumerator.NextEntityIndex(out var i))
        {
            float3 newValue = exampleComponents[i].Value + 1;
            exampleComponents[i] = newValue;
        }
    }
}
\end{lstlisting}
Wie man sieht, ist der \texttt{IJobChunk} Job deutlich länger und komplexer als der \texttt{IJobEntity} Job, obwohl beide das Gleiche machen. Jedoch kann man viel besser erkennen, was wirklich bei der Ausführung des Jobs passiert. Die \textit{Chunks}, welche getrennt im Speicher liegen, werden der Execute Funktion übergeben. Welche das sind, kann man mit einer \textit{EntityQuery} bestimmen, welche man bei dem \texttt{IJobEntity} Job nicht unbedingt benötigt. Diese kann so aussehen:
\begin{lstlisting}[style=code, label=EntityQuery]
EntityQuery query = GetEntityQuery(typeof(ExampleComponent));
\end{lstlisting}
Mit dieser \textit{EntityQuery} kann man dann den Job ausführen:
\begin{lstlisting}[style=code, label=JobExecution]
protected override void OnUpdate(){
    var job = new ExampleJob();
    job.exampleTypeHandle = GetComponentTypeHandle<ExampleComponent>(false);
    //Dependency muss mit dem Job aktualisiert werden	
    this.Dependency = job.ScheduleParallel(query, this.Dependency);
}
\end{lstlisting}
Zusätzlich braucht man, da man \textit{Components} auch deaktivieren kann, eine Bitmaske, welche verhindert, dass \textit{Entities} bearbeitet werden, deren \textit{Components} deaktiviert sind. Ein \texttt{IJobEntity} generiert den \texttt{IJobChunk} so, dass automatisch darauf geachtet wird. Deshalb wird empfohlen den Job mit einem \textit{Entity} zu nutzen, welcher einfacher zu implementieren ist.
\subsection{Burst Compiler} \label{burst}
Der Burst Compiler dient der Performance Optimierung. Er übersetzt den .NET Code zu optimierterem CPU Code welcher mit dem LLVM Compiler\footnote{https://llvm.org/} ausgeführt wird. Burst wird meist in Verbindung mit dem Job System von Unity verwendet. Dabei kann man vor allem Jobs, aber auch in ECS ganze Systeme mit Burst kompilieren. Damit etwas kompiliert wird nutzt man das \texttt{[BurstCompile]} Attribut über Jobs, oder Funktionen von Systemen:
\begin{lstlisting}[style=code, caption={BurstCompile Attribut, um Burst zu verwenden}]
//BurstCompile Attribut
[BurstCompile]
private struct ExampleJob : IJobEntity{}
\end{lstlisting}
Burst hat jedoch einige Einschränkungen welche Typen und Eigenschaften unterstützt werden und kann deshalb nicht überall eingesetzt werden. Vor allem mit verwalteten Datentypen kann Burst kaum umgehen, weshalb man Funktionen eines verwalteten System nicht mit Burst kompilieren kann.
\subsubsection{Burst Kompilierung}
Burst hat zwei Arten wie es den Code kompiliert:\\
1. just-in-time Kompilierung: Diese Methode wird in dem Unity Editor verwendet. Das heißt der Compiler kompiliert den Code dann wenn er verwendet wird. Das bedeutet auch, dass der Code zunächst mit dem normalen Mono Compiler\footnote{https://docs.unity3d.com/Manual/Mono.html} läuft, bis Burst im Hintergrund den Code kompiliert hat. Das heißt es wird asynchron kompiliert. Man kann jedoch auch Unity, mit dem Attribut \texttt{CompileSynchronously}, dazu zwingen den Code vor dem ersten Ausführen mit Burst zu kompilieren. Dies ist beispielsweise für Laufzeitanalysen sehr vorteilhaft.\\
2. ahead-of-time Kompilierung: Diese Methode wird bei dem bauen des Spiels verwendet. Bauen meint hierbei das Spiel von dem Unity Editor in ein ausführbares Programm umzuwandeln. Dabei speichert Burst den kompilierten Code in eine Bibliothek, welche mit dem Spiel ausgeliefert wird. Zur Laufzeit wird dann dieser Code verwendet.
\subsubsection{Beispiele}