\section{Einleitung}
Die Videospielbranche ist eine der größten Marktzweige die es auf der Welt gibt. Laut dem Marktforschungsunternehmen Newzoo hatte die Gaming-Industrie 2021 einen Gesamtumsatz von rund 192,7 Milliarden Dollar weltweit \cite{Spiele-Industrie-weltweit}. Dies ist mehr Umsatz, als Filme-, Serien- und aufgenommene Musikindustrie zusammengenommen. Auch in Deutschland liegt der Umsatz der Gaming-Industrie knapp vor dem Umsatz des Videostreamings und der Musik \cite{Spiele-Industrie-deutschland}.\\
Die Videospiele, welche heutzutage auf den Markt kommen, werden immer komplexer. Es werden immer größere Spiele entwickelt, welche mit immer mehr Inhalt gefüllt werden. Diese müssen aber natürlich trotzdem, oder vielleicht gerade deshalb auch Leistungsstandards entsprechen. Um diesen Anforderungen gerecht zu werden, sind fortschrittliche Programmieransätze erforderlich, die die Performance-Steigerung in komplexen Videospielen ermöglichen. Zwar steigt die Rechenleistung in modernen Computern weiter an, es passiert aber dennoch immer wieder, dass Spiele mit einem Objektorientierten Ansatz an ihre Grenzen stoßen, die Rechenleistung vollständig umzusetzen. Auch ist es sinnvoll, Videospiele energieeffizienter zu gestalten um Stromeinsparungen und Nachhaltigkeit zu erhalten. Das heißt, es wird in Zukunft wesentlich wichtiger sein, einen alternativen Lösungsansatz anzustreben, statt die immer größer werdende Rechenleistung auszunutzen. Dabei können ein Datenorientierter Ansatz bei der Programmierung und Parallelisierung des Programms eine Lösung sein.\\
Die vorliegende Arbeit geht dabei insbesondere auf den Datenorientierten Ansatz von Unity, also das \textit{Entity Component System} (ECS), ein. Die Spiele-Engine Unity wurde aus den folgenden Gründen gewählt:\\
1. Unity ist eine Spiele-Engine, welche führend in der Spiele-Industrie ist. Mit ihr wurden schon viele große Spiele entwickelt\footnote{https://www.thegamer.com/unity-game-engine-great-games/}.\\
2. Das ECS von Unity befindet sich momentan in großer Entwicklung. Unity setzt viel daran, den Datenorientierten Technologie-Stack auszubauen und es kamen viele Neuerungen in den letzten Monaten\footnote{https://unity.com/de/roadmap/unity-platform/dots}.