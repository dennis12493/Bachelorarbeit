\section{Einleitung}
Die Videospielbranche ist einer der umsatzstärksten Marktzweige der Welt. Laut dem Marktforschungsunternehmen Newzoo hatte die Gaming-Industrie 2021 einen Gesamtumsatz von rund 192,7 Milliarden Dollar weltweit. Dies ist mehr Umsatz, als Filme-, Serien- und aufgenommene Musikindustrie zusammengenommen \cite{Spiele-Industrie-weltweit}. Auch in Deutschland liegt der Umsatz der Gaming-Industrie knapp vor dem Umsatz des Videostreamings und der Musik \cite{Spiele-Industrie-deutschland}.\\
Videospiele, welche heutzutage auf den Markt kommen, werden immer komplexer. Es werden immer größere Spiele entwickelt, welche mit immer mehr Inhalt gefüllt werden. Diese müssen aber natürlich trotzdem, oder vielleicht auch gerade deshalb Leistungsstandards entsprechen. Um diesen Anforderungen gerecht zu werden, sind fortschrittliche Programmieransätze erforderlich, die eine Leistungssteigerung in komplexen Videospielen ermöglichen. Zwar steigt die Rechenleistung in modernen Computern weiter an, dennoch passiert es aber immer wieder, dass Spiele mit einem objektorientierten Ansatz an ihre Grenzen stoßen, die Rechenleistung vollständig zu nutzen. Auch ist es sinnvoll, Videospiele energieeffizienter zu gestalten um Stromeinsparungen zu erhalten und nachhaltiger zu werden. Das heißt, es wird in Zukunft wesentlich wichtiger sein, einen alternativen Lösungsansatz anzustreben, statt die immer größer werdende Rechenleistung auszunutzen. Dabei können ein datenorientierter Ansatz bei der Programmierung und eine Parallelisierung des Programms mögliche Lösungsansätze sein.\\
Die vorliegende Arbeit geht insbesondere auf den datenorientierten Ansatz von Unity, das \textit{Entity Component System} (\textit{ECS}), ein. Die Spiele-Engine Unity wurde aus den folgenden Gründen gewählt:\\
1. Unity ist eine Spiele-Engine, welche führend in der Gaming-Industrie ist. Mit ihr wurden schon viele große Spiele entwickelt\footnote{https://www.thegamer.com/unity-game-engine-great-games/}.\\
2. Das \textit{ECS} von Unity wird momentan aktiv weiterentwickelt. Unity setzt viel daran, den \textit{Datenorientierten Technologie-Stack} auszubauen und es kamen viele Neuerungen in den letzten Monaten auf den Markt\footnote{https://unity.com/de/roadmap/unity-platform/dots}.\\
Im Verlauf der Arbeit werden Eigennamen und englische Begriffe \textit{kursiv} geschrieben. Begriffe, die sich auf den Programmtext beziehen oder Variablen, Funktion etc. beschreiben, werden im \texttt{Typewriter} Format im Fließtext dargestellt.