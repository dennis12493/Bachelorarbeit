\section{Datenorientierte Programmierung}
\subsection{Was ist das Datenorientierte Design?}
Data-Oriented Design \cite{Data-OrientedDesign}\\
Probleme mit OOP: Hauptspeicher Zugriff und Cache misses. Versuche Code zu Parallelisieren sind zu viel Aufwand und bringen kaum etwas. Code ist sehr komplex.\\Datenorientiertes Design ist ein anderer Ansatz, der diese Probleme zu lösen versucht. Datenorientierung ändert die Sichtweise des Programmierens: Weg von Objekten, hin zu den eigentlichen Daten, wie diese im Speicher liegen und wie sie gelesen und verändert werden. Beim Programmieren geht es immer um das Verändern von Daten. Es ist die Beschreibung wie aus eingegebenen Daten veränderte Daten werden. Daher ergibt es Sinn sich direkt mit den Daten zu befassen. Zusätzlich: \glqq data-oriented design\grqq{}  hat nichts mit \glqq data-driven\grqq{} zu tun.\\ Ideale Daten:\\Kommt auf die Daten an und wie sie genutzt werden. Am besten wenn man die Daten mit möglichst geringem Aufwand nutzen kann. Also kleinstmögliche Veränderung. Das Programms wird um die ideale Datenstruktur gebaut.\\Objekte sind oft wie Bäume gebaut. Objekte interagieren oft mit anderen Objekte \glqq unter\grqq{} ihnen. Iteriert man über eine Anzahl an Objekten passiert das mehrfach mit beliebigen Objekten. Für ein Ideales Layout sollte ein Objekt in Komponenten zerlegt werden. Komponenten der gleichen Art können dann als Gruppe zusammen im Speicher liegen, egal von welchem Objekt sie kommen. Daraus resultierten große Gruppe homogener Daten, welche dann sequenziell verarbeitet werden können.\\Vorteile:\\Parallelisierung: Die Daten können sehr leicht auf mehrere Threads aufgeteilt werden ohne großen Aufwand\\Cache Affinität: Sehr Effizient, da der selbe Code immer wieder ausgeführt wird. Wenn die Daten sequenziell verarbeitet werden resultiert das in sehr guter Performance und fast perfekter Cache Nutzung.\\Modularität: Wenn Code zur Verbesserung der Performance angepasst wird, resultiert das oft in schlechter lesbarem und schlechter wartbarem Code. Bei Konzentration auf die Transformierung der Daten hat man am Ende kleinere Funktionen mit weniger Abhängigkeiten.\\Testing: Unit Test für Objektinteraktionen können kompliziert sein. Im Datenorientierten Design sind Unit Tests jedoch sehr einfach. Eingabe Daten erstellen, Funktion aufrufen und die Ausgabedaten verifizieren.\\Nachteile: Nicht die Lösung für alles. Schwierig zu lernen, da es ganz anders ist. Auch schwierig mit bestehendem prozedurale / Objektorientiertem Code zu verbinden.\\Anwendung: Klassifizierung, wie Daten genutzt werden: read-only / read-write / write-only. Welche Daten werden von dem System gebraucht? Nicht wie verhält sich ein Gegner sonder eher wie sich alle verhalten.\\Platz für OOP?: Teils / teils. Für einzelne Objekte (beispielsweise GUI) kann es sinnvoll sein. Sollte aber dennoch Datenorientiert geschrieben werden.
\\ DOD \cite{DOD}\\
Datenorientiertes Design kann auch mit anderen Programmier Paradigmen co-existieren. Datenorientierung wird mehr gebraucht, da Spiele immer komplexer werden und die Abstraktion durch OOP das Bottleneck sein wird. Überall sind Daten: Graphik auf dem Bildschirm, Positionen und Bewegung von Partikeln und so weiter. All diese Daten müssen auf etwas laufen, sei es eine VM oder etwas konkretes wie der CPU oder der GPU. Diese Daten existieren auf der Hardware irgendwo. Datenorientiertes Design ist das designen von Software, welche Transformationen auf wohldefinierten Daten ausführt. \\
Ein einfaches Beispiel:\\Angenommen man hat im Objektorientierten Personen Objekte, welche Attribute haben, wie Alter, Beruf und Geschlecht. Dann würden diese Daten im datenorientierten beispielsweise in einer Liste gespeichert werden. Wenn man jetzt das Durchschnittsalter von allen vorhandenen Personen errechnen will muss man im Objektorientierten alle Objekte einzeln ansprechen und dort das Alter abfragen und diese dann verrechnen. Im datenorientierten kann man sich aber direkt die Liste von allen Personen hernehmen. Somit ist der Zugriff und die Berechnung des Durchschnittsalter wesentlicher schneller.
\subsection{Umsetzung Datenorientierter Programmierung}