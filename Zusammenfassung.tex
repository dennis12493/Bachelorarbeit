\thispagestyle{empty}
\section*{Zusammenfassung}
Diese Arbeit befasst sich mit der Performancesteigerung in Videospielen durch datenorientierte Programmierung. Das datenorientierte Design ist noch sehr modern. Erst Ende der 2010er Jahre wurde das datenorientierte Design speziell bei Spielekonsolen populär\footnote{https://de.wikipedia.org/wiki/Datenorientiertes\_Design}. Mit einem datenorientierten Ansatz versucht man Speicherlokalität zu erreichen, so dass eine schnellere Verarbeitung der Daten möglich ist. Der Fokus wird auf die Art und Weise, wie Daten im Speicher liegen, sie gelesen, geschrieben und verarbeitet werden gelegt. Eine mögliche Implementierung ist dabei der datenorientierte Technologie-Stack von Unity. In dieser Arbeit wird die Leistungssteigerung betrachtet, die ein datenorientierter Ansatz bietet. Dazu wurde eine Spielesimulation in Unity objektorientiert und datenorientiert entwickelt, welche anschließend gebenchmarkt wurde. Die Benchmarks ergaben, dass durch einen datenorientierten Ansatz, eine sehr große Performancesteigerung möglich ist. Dies ist aber nicht für alle Anwendungen der Fall.

%Zunächst wird eine allgemeine Einführung in die Datenorientierte Programmierung gegeben. Anschließend wird der Datenorientierte Ansatz von Unity, also dem Datenorientierten Technologie-Stack erläutert. Dieser besteht aus dem \textit{Entity Component System} von Unity, einem Datenorientierten Framework, dem Unity Job System und dem Burst Compiler von Unity. Bei dem \textit{Entity Component System} wird besonders Wert auf die Speicherung und die Verarbeitung der Daten gelegt. Das Job System erlaubt es leicht parallelen Code zu schreiben und der Burst Compiler optimiert mit einigen Einschränkungen den entwickelten Code noch weiter. Zusätzlich wird ein eigens entwickeltes Videospiel vorgestellt, das sowohl Datenorientiert, als auch Objektorientiert umgesetzt wurde. Dabei wurde bei der Datenorientierten Implementierung alle Komponenten des Datenorientierten Technologie-Stacks eingesetzt. Damit wird dann in einem direkten Vergleich die Steigerung der Performance deutlich. Auch wird gezeigt in welchen Fällen ein Datenorientierter Ansatz sich besonders lohnt und wann ein Objektorientierter Ansatz vielleicht besser sein könnte.