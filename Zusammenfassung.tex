\thispagestyle{empty}
\section*{Zusammenfassung}
Die vorliegende Arbeit befasst sich mit der Performancesteigerung in Videospielen durch Datenorientierte Programmierung. Mit dem datenorientierten Ansatz wird versucht, die objektorientierte, nicht Cache freundliche Verarbeitung von Daten zu ersetzen. Der Fokus wird auf die Art und Weise gelegt, wie Daten im Speicher liegen, gelesen, geschrieben und verarbeitet werden. Hierdurch soll eine schnellere Verarbeitung der Daten erreicht werden. Eine mögliche Implementierung ist dabei der \textit{Datenorientierte Technologie-Stack} von Unity. In dieser Arbeit wird die Leistungssteigerung betrachtet, die ein datenorientierter Ansatz bietet. Dazu wurde eine Spielesimulation in Unity einmal objektorientiert und einmal datenorientiert entwickelt, welche anschließend gebenchmarkt wurde. Die Umsetzung des datenorientierten Spiels gestaltete sich schwieriger, als die des objektorientierten Spiels. Die Benchmarks ergaben, dass durch einen datenorientierten Ansatz eine enorme Performancesteigerung möglich ist.